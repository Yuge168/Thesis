\subsection{Monte Carlo Techniques for Solving Partial Differential Equations}

Another experimental method, Monte Carlo method, provides the dirct
statistical estimates of the solution to the PDEs. Monte Carlo methods
(MCMs), the commonly used computational techniques, aim to generate
samples from a given probability distribution, estimate the functions'
expectations under this distribution, and optimize the complicated
objective functions by using random numbers \cite{kroese2014monte}
\cite{rubinstein2016simulation}. MCMs can be used to solve the IBVPs
by generating the random numbers to simulate the successive positions
of the trajectory of a stochastic process at fixed instants
\cite{kronberg1976solution}\cite{king1951monte}, since the original
continuous problem can be represented by the probabilistic
interpretation and the solution can be approximated by the expectation
of some functional of the trajectories of a stochastic process
\cite{grebenkov2014efficient}\cite{sabelfeld2013random}. Therefore,
unlike the numerical techniques proposed in the
subsubsection~\ref{numerical methods}, the nondeterministic Monte
Carlo simulations are grid-free on the domain, boundary, and the
boundary conditions of the problem
\cite{grebenkov2014efficient}. Moreover, since MCMs are essentially
based on the statistical sampling, it has the inherent possible source
of errors result from the statistical fluctuations.


\subsubsection{Conventional Monte Carlo Methods}

In subsection~\ref{numerical_methdos}, the probabilistic
interpretation of Eq.~\ref{eq:final_FDM_heat_equation} is a free
Brownian motion particle randomly moving one step to $(i-1, j)$,
$(i+1, j)$, $(i, j-1)$, and $(i, j+1)$ with probabilities $\beta$ or
remaining at $(i, j)$ with probability $(1-4\beta)$. Monte Carlo
method is a straightfoward way to determine which direction the
particle should move by generating the random numbers. For example, 






\subsubsection{Continuous Monte Carlo Methods}


MCMs have been applied frequently in solving the elliptic partial
differential equations, for example, the Laplace’s equation and
Poisson’s equation \cite{haji1967application} \cite{booth1981exact}
\cite{muller1956some}. Unlike some studies initially replace the
domain of the problem by a network of points, generate the difference
approximation for the derivatives, and use MCMs to solve the
finite-difference equations, some continuous MCMs are developed for
approximating the solution of the elliptic partial differential
equations. The essential idea of those continous MCMs is to explore the
domain as fast as possible.  



\subsubsection{Discrete Monte Carlo Methods}

