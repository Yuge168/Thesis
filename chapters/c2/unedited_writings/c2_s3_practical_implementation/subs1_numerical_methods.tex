\subsection{Numerical Methods for Solving Parabolic Partial Differential Equations}\label{numerical_methdos}


The heat equation is a critical time-dependent parabolic partial
differential equation characterizing how a quantity diffuses through a
given region over time. From the physical interpretation of the heat
equation, its solution describes the heat distribution or temperature
varying in time and positions and can be obtained uniquely by
considering specific initial and boundary conditions. As described in
the section~\ref{traditinal_heat_content}, this thesis aims to
calculate the asymptotic expansion of the heat content, defined as the
integration of the solution over the space-dimension, for the shape
description of a bounded domain.  The general solution of the heat
equation is in one of the two standard forms
\cite{crank1979mathematics}. One is constituted of a series of error
functions or related integrals, which is most suitable for evaluating
short-time diffusion behaviour numerically. Another is in the form of
a trigonometrical series, which converges rapidly for a long time. If
the heat equation is defined in a cylinder, a series of Bessel
functions will replace the trigonometrical series.

However, the traditional analytical techniques for solving the heat
equation has many restrictions, and its applications to practical
problems will exhibit difficulties. Firstly, the numerical evaluation
of the analytical solutions is usually by no means trivial because
they are in the form of infinite series. Secondly, either irregular
geometries or discontinuities lead to the complexities, so the
explicit algebraic solutions are close to non-existed. Thirdly, the
purely analytical techniques can apply strictly only to the linear
form of the boundary conditions and to constant diffusion properties
\cite{crank1979mathematics}.

Therefore, numerical methods and computer simulations are more helpful
and applicable to find solutions to the partial differential equations
(PDEs) than calculating pure analytical solutions. The techniques for
solving initial-boundary value problems (IBVPs) based on numerical
approximations have existed for a long time and been developed
considerably including the finite-difference method (FDM), finite
element method (FEM), finite volume method (FVM), boundary element
method (BEM), and so forth.




\subsubsection{Finite Difference Method}

FDM is frequently utilized to converting the heat equation into a
system of algebraically solvable equations
\cite{grossmann2007numerical}. The basic idea is to replace the
derivatives in the equation by the difference quotients. For example,
the FTCS (Forward Time Centered Space) scheme
\cite{pletcher2012computational} discretizes the Laplace operator in
space and the time derivative, and then implements the boundary
conditions on the staggered grid for representing the original
continuous problem.


Let $u(x, y, t)$ be the heat distribution at position $(x, y)$ and
time $t$ in a $2-$dimensional homogeneous and isotropic domain $\Omega$. It is
well-known that without any internal heat sources in the domain, $u(x,
y)$ satisfies the heat equation

\begin{equation}\label{eq:Cartesian_heat_equation}
  u_t = D (u_{xx} + u_{yy})
\end{equation}

Note, $D$ is a constant diffusion coefficient and $u_t$ inidicates
partial derivative with respect to time $t$, while $u_{xx}$ and
$u_{yy}$ indicate second partial derivative with respect to $x$ and
$y$ repectively.

Before the implementation of FTCS, let descrtize $\Omega$ along the
$x-$axis and $y-$axis as a regular lattice. In other words, both the
range of $x$ and that of $y$ are divided into equal intervals
$\triangle l$. Also, the time is devided into equal interval
$\delta$. Let the corrdinates of a representative grid point $(x, y,
t)$ be $(i \triangle l, j \triangle l, n \delta)$, where $\triangle l$
is the distance between two neighboring sites of the lattice and
$\delta$ is the time step. For simplicity, we denote the value of $u$
at the point $(i \triangle l, j \triangle l)$ at time $n \delta$ by
$u(i, j, n)$.

The difference formula for time derivative is

\begin{equation}\label{eq:time_difference}
  u_t = \frac{u(i, j, n+1) - u(i, j, n)}{\delta} + \mathcal{O}(\delta)
\end{equation}

The difference formula for the spatial derivaive of $x$ and $y$ are

\begin{align}
  u_{xx} = \frac{u(i-1, j, n) - 2 u(i, j, n) + u(i+1, j, n)}{(\triangle l)^2} + \mathcal{O}((\triangle l)^2) \label{eq:x_difference} \\
  u_{yy} = \frac{u(i, j-1, n) - 2 u(i, j, n) + u(i, j+1, n)}{(\triangle l)^2} + \mathcal{O}((\triangle l)^2) \label{eq:y_difference}
\end{align}

Dropping the error terms $\mathcal{O}(\delta)$ and
$\mathcal{O}((\triangle l)^2)$ and substituting the
Eq.~\ref{eq:time_difference}, Eq.~\ref{eq:x_difference}, and
Eq.~\ref{eq:y_difference} into original heat equation
Eq.~\ref{eq:Cartesian_heat_equation}, there will have

\begin{equation}\label{eq:FDM_heat_equation}
  \frac{u(i, j, n+1) - u(i, j, n)}{\delta} = D(\frac{u(i-1, j, n) + u(i+1, j, n) - 4u(i, j, n) + u(i, j-1, n) + u(i, j+1, n)}{(\triangle l)^2})
\end{equation}

Rearranged Eq.~\ref{eq:FDM_heat_equation} as

\begin{equation}\label{eq:rearrange_FDM}
  u(i, j, n+1) = \frac{D\delta}{(\triangle l)^2} (u(i-1, j, n) - 2 u(i, j, n) + u(i+1, j, n) + u(i, j-1, n) - 2 u(i, j, n) + u(i, j+1, n)) + u(i, j, n)
\end{equation}

Finally, the value of $u(i, j, n+1)$ can be expressed explicitly in
terms of $u(i-1, j, n)$, $u(i+1, j, n)$, $u(i, j-1, n)$, $u(i, j+1, n)$, and $u(i, j, n)$ by

\begin{align}\label{eq:final_FDM_heat_equation}
  u(i, j, n+1) &= \beta (u(i-1, j, n) + u(i+1, j, n) + u(i, j-1, n) + u(i, j+1, n)) \\
               & + (1-4\beta) u(i, j, n) \\
  \beta &= \frac{D\delta}{(\triangle l)^2}
\end{align}


The FTCS is conditionally stable \cite{pletcher2012computational}
because the explicit formula in Eq.~\ref{eq:final_FDM_heat_equation}
is stable if and only if $\beta \leq \frac{1}{2}$, which means

\begin{equation}\label{eq:stable_condition}
  \delta \leq \frac{(\triangle l)^2}{2D}
\end{equation}

Eq.~\ref{eq:stable_condition} implies that if the spatial resolution
$\triangle l$ becomes doubled, the time-step $\delta$ should be
reduced by a factor of four to maintain the numerical stability, which
causes the extremely tiny time-step in the high-resolution
calculations. Moreover, there are three kinds of errors needed to be
considered when using FDM. First of all, in the derivation of the
finite-difference equations, the higher-order terms in the Taylor
series are neglected, constituting the truncation error. If the time
and space interval tends to $0$, the truncation errors will approach
$0$, or the FDM is incompatible or inconsistent with the original heat
equation \cite{crank1979mathematics}. Another class of error appearing
in FDM, called round-off error, results from the loss of precision due
to the computer rounding of decimal
quantities. \cite{hoffman2018numerical}. The last type of error is the
discretization error, which can be reduced by decreasing the time
size, grid size, or both \cite{crank1979mathematics}. Moreover, DFM
becomes less accurate and hard to implement when the problem is
defined in the irregular geometries since the heat equation must be
transformed before applying the Taylor series.





\subsubsection{Finite Element Method}


Unlike the FDM, the finite element method
(FEM) \cite{zlamal1968finite} divides the complicated geometries,
irregular shapes, and boundaries into the union of smaller and simpler
subdomains (eg. lattice, triangle, curvilinear polygons, etc.), which
are called finite elements \cite{logan2011first}. The smaller size of
the finite element mesh, the more accurate the approximate
solution. Therefore, FEM has great flexibilites or
adaptivities \cite{reddy1993introduction}. For example, FEM can
provide higher fidelity or higher accuracy in a local region and keep
elsewhere the same. Each subdomain is locally represented by the
element equation, continuous piecewise shape functions, which are
finally assembled into a larger system of algebraic equations for
modelling the entire problem. FEM aims to approximate the numerical
solution by minimizing the associated error function to meet certain
specification of the accuracy, which can be done by the
parallelization. Nevertheless, FEM heavily relies on the numerical
integration, where the quadrature rules sometimes cause
difficulties. FEM requires an amount of human involvement in the
process of building the FE model, checking the result, detecting and
updating the model design. Moreover, compared with FDM, FEM demands a
longer execution time and a larger amount of input data.



\subsubsection{Other Numerical Techniques}


Another method closely related to the FEM is the finite volume method
(FVM). It converts the original heat equation into the integral
forms \cite{eymard2000finite}. However, the accuracy of FVM is related
to the integration with respect to the time and space. Unlike the
domain-type methods (e.g. FDM, FEM, FVM, etc.), the boundary element
method (BEM) transforms the heat equation, defined in a given domain,
into an integral equation over the boundary of the domain using the
boundary integral equation
method \cite{attaway1991boundary}. Especially, when the domain extends
to infinity or the boundary is complex, BEM is more efficient in
computation than other methods because of the smaller surface or
volume ratio \cite{katsikadelis2002boundary} since it only discretizes
the boundary and fits the boundary values into the integral
equation \cite{ang2007beginner}. However, the matrics resulted in BEM
are generally unsymmetric and fully populated, which are difficult to
be solved \cite{mushtaq2010advantages}.


\subsubsection{Limitations in Practice}



In this thesis, the heat equation defined in $2$-dimensional domain,
which is bounded by the border of the image and the whole root system,
with millions of pixels, the extremely complex roots and various
boundary conditions. Before the calculation of the heat content
contained in the domain, the numerical computational techniques can be
used to approximate the solutions of the heat equation, but some
practical difficulities have to be considered since all the described
numerical methods have an intrinsically similar feature - mesh
discretization in the time and space dimension. For instance, the far
more efforts are required in sovling heat equation by FDM and FVM
because of the complicated boundary of the roots and non-continuous
issues. Although the whole $2$-dimensional root image can be regarded
as a discretized domain, it is still time-consuming and challenging to
trace and identify the boundary of roots, label the nodes, and
generate the coordinates and connectivities among the nodes in the
preprocessing stage of FEM. The finer discrtization, the more accurate
approximation of the original IBVP in the numerical methods. More
importantly, the heat content defined as the integration of the
numerical solution of the heat equation over the space dimension
should also be approximated numerically, which results in the extra
effort and errors.
