\subsection{Output Analysis}

The output of the fixed-time step Monte Carlo simulations are particles' first passage time $t$, which is the number of steps taken by the particles hitting any positions of the target for the first time. Since first-hitting-time models are a sub-class of survival analysis in statistics \cite{altman1990practical}, it is straightforwad to use the Kaplan-Meier estimator to estimate the survival function $S(t)$ \cite{kleinbaum2005competing} of the numerical simulation, which provide the probability that a particle remains wandering beyound any specified time. Moreover, the pointwise upper and lower confidence interval can also calculatated by the Greenwood’s exponential formula \cite{hosmer2011applied}. In this subsection, the Kaplan-Meier estimator and confidence interval of $S(t)$ are introduced theoretically. However, in practice, the existed Python module, lifeline \cite{davidson2019lifelines}, will be used to implement the estimation. After obtaining the estimated survival function of the fixed-time step Monte Carlo simulations, the scaling relationship between $t$ and $\tau$ is derived for the valiation of research methodology in the next chapter.
