\subsubsection{Numerical Methods}\label{numerical methods}

The techniques for solving initial-boundary value problems (IBVPs) based on numerical approximations have existed for a long time. Recently, the considerable developments have been taken place in the numerical analysis and constructing the efficient computer programs to get the numerical solutions. 


The finite-difference methods (FDM) are frequently used and easily
implemented in solving the heat equation by converting it into a
system of algebraically solvable equations
\cite{grossmann2007numerical}. The basic idea is to replace the
derivatives in the equations by difference quotients. For example, the
FTCS (Forward Time Centered Space) scheme
\cite{pletcher2012computational} aims to discretize the Laplace
operator in space and the time derivative and implement the boundary
conditions on the staggered grid to represent the original continuous
problem, but it is conditionally stable
\cite{pletcher2012computational}. As the condition stats, a doubling
of the spatial resolution requires a simultaneous reduction in the
time-step by a factor of four in order to maintain the numerical
stability, but it causes the extremly tiny time-step in the high
resolution calculations. There are three kinds of errors needed to be
considered in using FDM. First of all, in the derivation of the
finite-difference equation, the higher-order terms in the Taylor
series are neglected, which constitue a truncation error. If the time
and space interval tends to $0$, the truncation error will approchs
$0$, or the FDM is incompatible or inconsistent with the heat equation
\cite{crank1979mathematics}. Another class of error appearing in FDM
called round-off error, which results from the loss of precision due
to the computer rounding of decimal quantities.
\cite{hoffman2018numerical}. The last class of error is the discritazation
error, the difference between the exact solutions of the original heat
equation and that of the approximating difference equation, which can
be reduced by decreasing the time size, grid size, or both
\cite{crank1979mathematics}. More importantly, for the irregularly shaped geometries, the heat equation must be transformed before the Taylor series can be applied, which causes problems in terms of additional cross-coupling of equations, mesh generation and general convergence.


Unlike the FDM, the finite element method
(FEM) \cite{zlamal1968finite} divides the complicated geometries,
irregular shapes, and boundaries into the union of smaller and simpler
subdomains (eg. lattice, triangle, curvilinear polygons, etc.), which
are called finite elements \cite{logan2011first}. The smaller size of
the finite element mesh, the more accurate the approximate
solution. Therefore, FEM has great flexibilites or
adaptivities \cite{reddy1993introduction}. For example, FEM can
provide higher fidelity or higher accuracy in a local region and keep
elsewhere the same. Each subdomain is locally represented by the
element equation, continuous piecewise shape functions, which are
finally assembled into a larger system of algebraic equations for
modelling the entire problem. FEM aims to approximate the numerical
solution by minimizing the associated error function to meet certain
specification of the accuracy, which can be done by the
parallelization. Nevertheless, FEM heavily relies on the numerical
integration, where the quadrature rules sometimes cause
difficulties. FEM requires an amount of human involvement in the
process of building the FE model, checking the result, detecting and
updating the model design. Moreover, compared with FDM, FEM demands a
longer execution time and a larger amount of input data.


Another method closely related to the FEM is the finite volume method (FVM). It converts the original heat equation into the integral forms \cite{eymard2000finite}. However, the accuracy of FVM is related to the integration with respect to the time and space. Unlike the domain-type methods (e.g. FDM, FEM, FVM, etc.), the boundary element method (BEM) transforms the heat equation, defined in a given domain, into an integral equation over the boundary of the domain using the boundary integral equation method \cite{attaway1991boundary}. Especially, when the domain extends to infinity or the boundary is complex, BEM is more efficient in computation than other methods because of the smaller surface or volume ratio \cite{katsikadelis2002boundary} since it only discretizes the boundary and fits the boundary values into the integral equation \cite{ang2007beginner}. However, the matrics resulted in BEM are generally unsymmetric and fully populated, which are difficult to be solved \cite{mushtaq2010advantages}. 


In this thesis, the heat equation defined in $2$-dimensional domain, which is bounded by the border of the image and the whole root system, with millions of pixels, the extremely complex roots and various boundary conditions. Before the calculation of the heat content contained in the domain, the numerical computational techniques can be used to approximate the solutions of the heat equation, but some practical difficulities have to be considered since all the described numerical methods have an intrinsically similar feature - mesh discretization in the time and space dimension. For instance, the far more efforts are required in sovling heat equation by FDM and FVM because of the complicated boundary of the roots and non-continuous issues. Although the whole $2$-dimensional root image can be regarded as a discretized domain, it is still time-consuming and challenging to trace and identify the boundary of roots, label the nodes, and generate the coordinates and connectivities among the nodes in the preprocessing stage of FEM. The finer discrtization, the more accurate approximation of the original IBVP in the numerical methods. More importantly, the heat content defined as the integration of the numerical solution of the heat equation over the space dimension should also be approximated numerically, which results in the extra effort and errors. 





