\subsubsection{\highlight[id=Yuge, comment={The writing of this part has been revised. Dave, please give me the feedback or comments on it. }]{Numerical Methods}}\label{numerical methods}

The techniques for solving initial-boundary value problems (IBVPs)
based on numerical approximations have existed for a long time and
been developed considerably including the finite-difference method
(FDM), finite element method (FEM), finite volume method (FVM),
boundary element method (BEM), and so forth.


FDM is frequently utilized to converting the heat equation into a
system of algebraically solvable
equations \cite{grossmann2007numerical}. The basic idea is to replace
the derivatives in the equation by the difference quotients. For
example, the FTCS (Forward Time Centered Space)
scheme \cite{pletcher2012computational} discretize the Laplace
operator in space and the time derivative and implement the boundary
conditions on the staggered grid for representing the original
continuous problem, but it is conditionally
stable \cite{pletcher2012computational}. If the spatial resolution
becomes doubled, the time-step should be reduced by a factor of four
to maintain the numerical stability, which causes the extremely tiny
time-step in the high-resolution calculations. There are three kinds
of errors needed to be considered when using FDM. First of all, in the
derivation of the finite-difference equations, the higher-order terms
in the Taylor series are neglected, constituting the truncation
error. If the time and space interval tends to $0$, the truncation
errors will approach $0$, or the FDM is incompatible or inconsistent
with the original heat equation \cite{crank1979mathematics}. Another
class of error appearing in FDM, called round-off error, results from
the loss of precision due to the computer rounding of decimal
quantities. \cite{hoffman2018numerical}. The last type of error is the
discretization error, which can be reduced by decreasing the time
size, grid size, or both \cite{crank1979mathematics}. Moreover, DFM
becomes less accurate and hard to implement when the problem is
defined in the irregular geometries since the heat equation must be
transformed before applying the Taylor series.


Unlike the FDM, FEM \cite{zlamal1968finite} divides the complicated
and irregular geometries and boundaries into the union of smaller and
simpler subdomains or finite elements \cite{logan2011first},
e.g. lattice, triangle, curvilinear polygons, etc. Each subdomain is
locally represented by the element equation, continuous piecewise
shape functions, which are finally assembled into a larger system of
algebraic equations for modelling the entire problem. The numerical
solution can be obtained by minimizing the associated error function
to meet the certain specification of the accuracy. The smaller size of
the finite element mesh, the more accurate the approximate
solution. FEM has great flexibilities or
adaptivity \cite{reddy1993introduction}. For instance, FEM can provide
higher fidelity or accuracy in a specific local region and keep
elsewhere identical. Nevertheless, FEM requires an amount of human
involvement in building the FE model, checking the result, detecting
and updating the model design. Moreover, compared with FDM, FEM
demands a longer execution time and a larger amount of input data.


FVM, closely related to FEM, converts the original heat equation into
the integral forms \cite{eymard2000finite}. However, the accuracy of
FVM is related to the numerical integration over time and space
dimensions. Unlike the domain-type methods (e.g. FDM, FEM, FVM, etc.),
BEM transforms the heat equation into an integral equation over the
boundary of the domain using the boundary integral equation
method \cite{attaway1991boundary}. Especially, when the domain extends
to infinity or the boundary is complex, BEM is more efficient in
computation than other methods because of the smaller surface or
volume ratio \cite{katsikadelis2002boundary} since it only discretizes
the boundary and fits the boundary values into the integral
equation \cite{ang2007beginner}. However, it is arduous to solve the
matrics generated in BEM, since they are generally unsymmetric and
fully populated \cite{mushtaq2010advantages}.


In summary, all the described numerical methods have an intrinsically
similar feature - mesh discretization in the time and space
dimension. In this thesis, the heat equation is defined in a
$2$-dimensional domain, bounded by the border of the image and that of
the root system, with millions of pixels, the extremely complex roots
and various boundary conditions. It may be possible to calculate the
heat content contained in this domain by the numerical integration of
the solutions approximated by those numerical techniques. However,
some practical problems have to be taken into consideration. For
instance, the far more efforts are required when applying FDM and FVM
because of the complicated boundary of the roots and non-continuous
issues. Although the whole $2$-dimensional root image can be viewed as
a discretized domain, it is still time-consuming and challenging to
trace and identify the boundary of roots, label the nodes, and
generate the coordinates and connectivities among the nodes in the
preprocessing stage of FEM. The finer discretization, the more
accurate approximated solutions of the original IBVP and the longer
computational time spent by the numerical methods. More importantly,
the heat content, which is the integration of the numerical solution
over the space dimension, should also be approximated numerically
resulting in extra effort and errors.





