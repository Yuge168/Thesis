\subsubsection{Probabilistic Interpretation}\label{probabilistic interpretation}

In the subsection~\ref{analytical results}, the heat equation describes the temperature distribution of a certain homogeneous and isotropic domain that does not exist any heat sources \cite{varadhan1980lectures}, and its solution can be interpreted as the temperature changed over the position and time. Also, the heat equation and its solution can be interpreted by the Brownian motion \cite{brown1828microscopical} from a probabilistic perspective. Brownian motion, also called the Wiener process, is a stochastic process \cite{karlin2014first} defined in the continuous time and continuous space with the continuous sample paths and stationary independent increments \cite{ito2012diffusion}. This process also has the Markov property: the future state depends only on the present state \cite{bharucha2012elements}. In the probability theory, if a large number of free particles are considered undergoing the Brownian motion independently, then the density of particles at a specific time becomes a deterministic process called diffusion, which satisfies the heat equation \cite{kac1947random}\cite{varadhan1980lectures}. 



\paragraph{Survival Probability}

\newcommand{\cpu}{u(\hat r, \theta, \tau | \hat{r_0}, \theta_0, 0)}

For simplicity, the proabilistic interpretation of the problem as descirbed in the subsection~\ref{analytical results} is introduced. $\cpu$, the conditional probability of finding a Brownian motion particle at the position $(\hat r, \theta) \in \Omega$ at time $\tau>0$ started from $(\hat{r_0}, \theta_0) \in \Omega$ at $\tau=0$ without encountering the absorbing boundary, is governed by the heat equation Eq.~\ref{eq:polar_coordinate_diffusion}. It also satisfies the initial condition Eq.~\ref{eq:initial_cd}, the Dirichlet boundary condition Eq.~\ref{eq:Dirichlet_bc}, and the Neumann boundary condition Eq.~\ref{eq:Neumann_bc}. The "local" survival probability $S(\tau | \hat{r_0}, \theta_0, 0)$, represents the probability of finding a particle, localized at a position $(\hat{r_0}, \theta_0)$ at $\tau=0$, remains diffusing in the domain $\Omega$ at time $\tau > 0$ unabsorbed by the boundary $\Omega_1$, is

\begin{equation}\label{eq:cpu_s_local}
  S(\tau | \hat{r_0}, \theta_0, 0) = \int_{0}^{2\pi} d\theta \int_{1}^{\mu} \hat r \cpu d \hat r 
\end{equation}

As particals are distributed uniformly over the whole domain $\Omega$, the "global" survival probability $S(\tau)$, the parbability that a diffusing particle has not hit the absorbing boundary by the time $\tau$, can be expressed by

\begin{equation}\label{eq:cpu_s_global}
  S(\tau) = \frac{1}{|\Omega|}\iint_{\Omega} \hat{r_0} S(\tau | \hat{r_0}, \theta_0, 0) d \hat{r_0} d \theta_0
\end{equation}

Eq.~\ref{eq:cpu_s_local} and Eq.~\ref{eq:cpu_s_global} reveal that the heat content $Q_{\Omega}(\tau)$ expressed in Eq.~\ref{eq:annulus_analytical_s} is proportional to the survival probability $S(\tau)$ \cite{kalinay2011survival}. 


\paragraph{Mean First-Passage Time}

The first passage phenomena play a fundamental role in stochastic processes triggered by a first-passage event \cite{van1992stochastic},  which happens because of the presence of the absorbing boundary condition. One of the essential first-passage-related quantities in this thesis is the first-passage time or the first-hitting time \cite{redner2001guide}, which is the time taken by particle undergoing the Brownian motion from an inital position to any sites of $\Omega_1$ for the first time. Particles' mean first-passage time $\langle \tau \rangle$, also called the average first-passage time, implies an overall property of the annulus and has a closed relationship with the survival probability \cite{redner2001guide}

\begin{equation}\label{eq:mfpt}
  \begin{split}
    \langle \tau \rangle &= \int_{0}^{\infty} \tau dS\\
    &=\sum_{n=1}^{\infty} \frac{4}{\mu^2 - 1}
    \frac{1}{\lambda^2_{0,n}\bigg\{\bigg[\frac{J_0(\sqrt{\lambda_{0,n}})}{J'_0(\mu\sqrt{\lambda_{0,n}})}\bigg]^2
      -1\bigg\}}
  \end{split}
\end{equation}


\paragraph{Brownian Motion and Random Walks}

Brownian motion, the irregular motion of individual particles, has been existed for a long time before the random-walk theory. At the beginning of the twentieth century, the term, random walk, was initially proposed by Karl Pearson \cite{pearson1905problem}. He described a simple example of the isotropic planar random flights to model how mosquitoes migrate and invade randomly in the cleared jungle regions. At each time step, the mosquito moves to a random direction with a fixed step length. Rayleigh \cite{rayleigh1905problem} stated that Pearson's question, the distributions of mosquitos after many steps have been taken, is the same as superpositioning the sound vibrations with unit amplitude and arbitrary phase \cite{de2012flying}. At almost the same time, Louis Bachelier \cite{bachelier1900theorie} developed a model of the financial time series based on the random walk and explored the connection between discrete random walks and continuous heat equation. During the development of random
walk theory, many other important fields, such as random processes, random noise, spectral analysis, and stochastic equations, were developed by some physicists \cite{einstein1905electrodynamics} \cite{einstein1906theory} \cite{smoluchowski1916drei}. The continuous Brownian motion can be considered as the limit of discrete random walk as the time and space increments go to zero \cite{lawler2010random}\cite{varadhan1980lectures}.


\paragraph{Lattice Random Walks (LRWs)}

Let us consider a large amount of particles performing the simple random walk on the $d$-dimensional integer grid $\mathbb{Z}^d$. It is a discrete-space and discrete-time symmetric hopping process \cite{redner2001guide} on the lattice. At each time step, particle moves to one of its $2d$ nearest neighbours with probability $\frac{1}{2d}$. If $d \leq 2$, the random walk is recurrent \cite{hughes1998random}, which means the particle will returns to its origin infinitely often with the probability $1$. If $d \geq 3$, the random walk is transient \cite{hughes1998random}, that is, the particle will return to its origin only finitely often with probability $1$ \cite{hughes1998random}.


For simplicity, we start from considering $2$-dimensional lattice random walks (LRWs). Let $\triangle l$ be the distance between two sites and $\delta$ be the time step. Define $p(x,t| x_{S})$ as the conditional probability of a particle to be in position $x$ at time $t$ starting from $x_{S}$ at time $t=0$. $p(x, t + \delta |x_{S})$ is defined as probability of a particle to be in position $x$ at time $t + \delta$ starting from $x_{S}$ at time $t=0$, \cite{lawler2010random}


\begin{equation}\label{eq:lrws_next}
  p(x, t + \delta | x_{S}) = \frac{p(x - \triangle l e_{x}, t| x_{S}) +
    p(x + \triangle l e_{x}, t| x_{S}) + p(x - \triangle l e_{y}, t|
    x_{S}) + p(x + \triangle l e_{y}, t| x_{S})}{4}
\end{equation}

where $e_x$ and $e_y$ are unit vectors of $x-$ axis and $y-$ axis, respectively.


When $\delta \rightarrow 0$, $\triangle l \rightarrow 0$, and $\delta \sim (\triangle l)^2$, 

\begin{equation}\label{eq:lrws_prob}
  p(x, t | x_{S}) + \delta p(x, t | x_{S}) = p(x, t | x_{S}) +
  \frac{(\triangle l)^2}{4} (\frac{\partial ^2 p(x, t|
    x_{S})}{\partial x^2} + \frac{\partial^2 p (x, t| x_{S})}{\partial
    y^2})
\end{equation}

Finally, the $2$-dimensional heat equation is

\begin{equation}\label{eq:lrws_heat}
  \frac{\partial p(x, t| x_{S})}{\partial t} = \frac{(\triangle
    l)^2}{4 \delta} (\frac{\partial ^2 p(x, t| x_{S})}{\partial x^2} +
  \frac{\partial^2 p (x, t| x_{S})}{\partial y^2})
\end{equation}
where $D = \frac{(\triangle l)^2}{4 \delta}$ is the diffusion coefficent. This derivation shows a tight relationship between lattice random walks and diffusion.



\paragraph{Pearson's Random Walks (PRWs)}

Based on Pearson's problem and Rayleigh's answer, Stadje
\cite{stadje1987exact} and Masoliver et al. \cite{masoliver1993some} considered a two-dimensional continuous-time and continuous-space random walk, defined as Pearson's random walks (PRWs) in this thesis. In PRWs, particle moves with constant speed and with random directions distributed uniformly in $[0, 2\pi)$. Moreover, the lengths of the straight-line paths and the turn angles are stochastically independent. If the mean step length approaches zero and the walking time is big enough, the behaviours of particles in PRWs weakly converges to the Wiener Process \cite{stadje1987exact}, which satisfies the traditional heat equation.
