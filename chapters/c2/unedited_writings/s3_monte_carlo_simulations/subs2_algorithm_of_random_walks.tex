\subsection{Algorithms of Random Walks}

In the subsection~\ref{background}, the practical challenges of
solving the heat equation defined in the $2-$ dimensional domain with
the complicated root systems by numerical methods and Monte Carlo
simulations are revealed, and the probabilistic interpretation of the
heat equation, survival probability, and random walks are introduced.
From Eq.~\ref{eq:annulus_analytical_s}, Eq.~\ref{eq:cpu_s_global}, and
Eq.~\ref{eq:mfpt}, the final goal in this thesis is to approximate the
heat content, or the survival probability, defined as the integrals,
which only depends on the time variable. In other words, our interest
is only related to the time when the first-passage event happens in
the stochastic process, that is, the first-passage time. Therefore, in
this subsection, two random walks algorithms are designed to mimic
particles' first-passage time by $2-$ dimensional fixed-time step
Monte Carlo simulations in the real root images for approximating the
integration as expressed in Eq.~\ref{eq:cpu_s_global} and
Eq.~\ref{eq:mfpt}.


\subsubsection{Lattice Random Walks}


  \begin{algorithmic}
    \caption{Lattice Random Walks (LRWs)}\label{lrws_algorithm}
    \floatname{algorithm}{Procedure}
    \renewcommand{\algorithmicrequire}{\textbf{Input:}}
    \renewcommand{\algorithmicensure}{\textbf{Output:}}

  \end{algorithmic}






\subsubsection{Pearson's Random Walks}

\begin{algorithm}
  \caption{Pearson's Random Walks (PRWs)}\label{prws_algorithm}
  
\end{algorithm}
