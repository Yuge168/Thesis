\subsubsection{\highlight[id=Yuge, comment={This writing has been revised. Dave, please give me the feedback and comments. Thanks!}]{Probabilistic Interpretation}}\label{probabilistic interpretation}

In the subsection~\ref{analytical results}, the heat equation
describes the temperature distribution of a homogeneous and isotropic
domain \cite{varadhan1980lectures}, and its solution characterizes how
the temperature changes over the position and time. From the
probabilistic perspective, the heat equation and its solution can also
be understood by the Brownian
motion \cite{brown1828microscopical}. The Brownian motion also called
the Wiener process, is a continuous-time and continuous-space
stochastic process \cite{karlin2014first} with the continuous sample
paths and stationary independent
increments \cite{ito2012diffusion}. This process also has the Markov
property: the future state depends only on the present
state \cite{bharucha2012elements}. In the probability theory, if a
large number of free particles undergoing the Brownian motion
independently, the density of particles at a specific time becomes a
deterministic process called diffusion, which satisfies the heat
equation \cite{kac1947random}\cite{varadhan1980lectures}.



\paragraph{Survival Probability}


\newcommand{\prob}{P(\hat r, \theta, \tau | \hat{r_0}, \theta_0, 0)}
% Define the conditional probability u, which will be used frequently. 
                                 

For simplicity, we only investigate the probabilistic interpretation
of the heat equation defined in the annulus with the boundary and
initial conditions as same as described in the
subsection~\ref{analytical results}. Consider a particle undergoing
the Brownian motion from $(\hat{r_0}, \theta_0) \in \Omega$ at
$\tau=0$ and let $\prob$ be the conditional probability of finding the
particle at $(\hat r, \theta) \in \Omega$ at time $\tau>0$. $\prob$
satisfies the following equations

\begin{align}
  \frac{\partial \prob}{\partial \tau} &= \Delta \prob
  \qquad\text{for $(\hat r, \theta) \in \Omega$}\label{eq:polar_coordinate_diffusion_conditional_prob} \\
  \prob & = 0
  \qquad\text{for $(\hat r, \theta) \in \partial \Omega_1$}\label{eq:Dirichlet_bc_conditional_prob} \\
  \frac{\partial}{\partial \hat r} \prob &= 0 
  \qquad\text{for $(\hat r, \theta) \in \partial \Omega_2$}\label{eq:Neumann_bc_conditional_prob} \\
  \prob & = \frac{1}{\hat r}\delta(\hat r - \hat{r_0}) \delta(\theta - \theta_0)
  \qquad\text{for $\tau = 0$}\label{eq:initial_cd_conditional_prob}
\end{align}

Note, Eq.~\ref{eq:polar_coordinate_diffusion_conditional_prob} is the
heat equation and supplemented by the initial condition
Eq.~\ref{eq:initial_cd_conditional_prob} with the Dirac $\delta -$
distribution, absorbing boundary condition
Eq.~\ref{eq:Dirichlet_bc_conditional_prob}, and reflecting boundary
condition Eq.~\ref{eq:Neumann_bc_conditional_prob}. Similarly, the
conditional probability $\prob$ can be expressed in terms of the
eigenvalues and eigenfunctions. 

Define the local survival probability $S(\tau | \hat{r_0}, \theta_0,
0)$ as the probability that the particle, localized initially at
$(\hat{r_0}, \theta_0)$, keeps diffusing in $\Omega$ at time $\tau >
0$ without being absorbed by $\Omega_1$. It can be calculated by


\begin{equation}\label{eq:cpu_s_local}
  S(\tau | \hat{r_0}, \theta_0, 0) = \iint_{\Omega} \hat r \prob d \hat r d\theta
\end{equation}
  
Thus, the survival probability $S(\tau)$ for a particle, starting with
an initial distribution uniformly spread in $\Omega$, is the average
of the local survival probability over $\Omega$. It is

\begin{equation}\label{eq:cpu_s_global}
  S(\tau) = \frac{1}{|\Omega|}\iint_{\Omega} \hat{r_0} S(\tau | \hat{r_0}, \theta_0, 0) d \hat{r_0} d \theta_0
\end{equation}

Eq.~\ref{eq:cpu_s_local} and Eq.~\ref{eq:cpu_s_global} reveal that the
heat content $Q_{\Omega}(\tau)$ in Eq.~\ref{eq:annulus_analytical_s},
calculated as the integration of the temperature over $\Omega$, is
proportional to the survival probability
$S(\tau)$ \cite{kalinay2011survival}. 


\paragraph{Mean First-Passage Time}

The first passage phenomena play a fundamental role in the stochastic
processes triggered by a first-passage
event \cite{van1992stochastic}. In this thesis, the first-passage
event refers to the first time when the Brownian motion is stopped
because of the present of the Dirichlet boundary condition. One of the
essential first-passage-related quantities is the first-passage time
or the first-hitting time \cite{redner2001guide}, which is the time
taken by particle undergoing the Brownian motion from an initial
position to any sites of $\Omega_1$ for the first time. Particles'
mean first-passage time $\langle \tau \rangle$, also called the
average first-passage time, has a closed relationship with the
survival probability $S(\tau)$ \cite{redner2001guide}

\begin{equation}\label{eq:mfpt_conditional_prob}
  \begin{split}
    \langle \tau \rangle &= \int_{0}^{\infty} \tau dS(\tau)\\
    &=\sum_{n=1}^{\infty} \frac{4}{\mu^2 - 1}
    \frac{1}{\lambda^2_{0,n}\bigg\{\bigg[\frac{J_0(\sqrt{\lambda_{0,n}})}{J'_0(\mu\sqrt{\lambda_{0,n}})}\bigg]^2
      -1\bigg\}}
  \end{split}
\end{equation}


From Eq.~\ref{eq:mfpt_conditional_prob} and
Eq.~\ref{eq:eigenfunction}, it is clear that $\langle \tau \rangle$
implies an overall property of the annulus since it only depends on
the radius ratio of annulus $\mu$.


\paragraph{Brownian Motion and Random Walks}

Brownian motion, the irregular motion of individual particles, has
been existed for a long time before the random-walk theory was
developed. At the beginning of the twentieth century, the term, random
walk, was initially proposed by Karl
Pearson \cite{pearson1905problem}. He utilized the isotropic planar
random flights to model how mosquitoes migrate and invade randomly in
the cleared jungle regions. At each time step, the mosquito moves to a
random direction with a fixed step
length. Rayleigh \cite{rayleigh1905problem} answered Pearson's
question in the same year, that is, the distributions of mosquitos
after many steps have been taken is identical to superposition the
sound vibrations with unit amplitude and arbitrary
phase \cite{de2012flying}. At almost the same time, Louis
Bachelier \cite{bachelier1900theorie} designed a model for the
financial time series based on the random walks. Louis Bachelier also
explored the relationship between discrete random walks and the
continuous heat equation. During the development of random-walk
theory, many other scientific fields, including the random processes,
random noise, spectral analysis, and stochastic equations, were
developed by some
physicists \cite{einstein1905electrodynamics} \cite{einstein1906theory} \cite{smoluchowski1916drei}. The
continuous Brownian motion is the scaling limit of the discrete random
walks as the time and space increments approach
zero \cite{lawler2010random}\cite{varadhan1980lectures}.


\paragraph{Lattice Random Walks (LRWs)}

\newcommand{\p}{p(x,y,t)}


Let us consider a particle performing the simple random walk on the
$d$-dimensional integer grid $\mathbb{Z}^d$. It is a discrete-space
and discrete-time symmetric hopping process \cite{redner2001guide} on
the lattice. At each time step, the particle moves to one of its $2d$
nearest neighbours with probability $\frac{1}{2d}$. If $d \leq 2$, the
random walk is recurrent, which means the
particle will return to its origin infinitely often with the
probability $1$. If $d \geq 3$, the random walk is
transient, which indicates the particle will
return to its origin only finitely often with the probability
$1$ \cite{hughes1998random} \cite{lawler2010random}. Only $2$-dimensional lattice random walks
(LRWs) and its connection with the heat equation are introduced since
this thesis aims to explore and characterize the shape of roots in the
$2-$dimensional images.

Let $\triangle l$ be the distance between two sites in the lattice and
$\delta$ be the time step. Let $p(i, j, n)$ be the probability of
finding a particle to be in position $(i, j)$ after $n$ steps. There has

\begin{equation}\label{eq:p_ijn}
p(i, j, n) = \frac{p(i-\triangle l, j, n-1) + p(i + \triangle l, j, n-1) + p(i, j - \triangle l, n-1) + p(i, j + \triangle l, n-1)}{4}
\end{equation}

\begin{align}\label{eq:p_diff}
p(i, j, n) - p(i, j, n-1) & = \frac{1}{4}\left(p(i-\triangle l, j, n-1) - 2p(i,j, n-1) + p(i + \triangle l, j, n-1)\right) \notag \\
&\quad + \frac{1}{4}\left(p(i, j - \triangle l, n-1) - 2p(i,j, n-1) + p(i, j + \triangle l, n-1)\right)
\end{align}


\begin{align}
x &= i \triangle l \label{eq:x}\\
y &= j \triangle l \label{eq:y}\\
t &= n \delta \label{eq:t}
\end{align}

Eq.~\ref{eq:x} and Eq.~\ref{eq:y} can be used to definde particle's position in the $x-y$ coordinate system and Eq.~\ref{eq:t} is the walking time taken by the particle after $n$ steps.

\begin{align}
p(i, j, n) - p(i, j, n-1) &= p(x, y, \frac{t}{\delta}) - p(x, y, \frac{t-\delta}{\delta}) \\
& \approx \delta \frac{\partial \p}{\partial t} \label{eq:t_diff}
\end{align}

Similarly,

\begin{align}
p(i-\triangle l, j, n-1) - 2p(i,j, n-1) + p(i + \triangle l, j, n-1) & \approx (\triangle l)^2 \frac{\partial^2 \p}{\partial x^2}\label{eq:x_diff}\\
p(i, j - \triangle l, n-1) - 2p(i,j, n-1) + p(i, j + \triangle l, n-1) & \approx (\triangle l)^2 \frac{\partial^2 \p}{\partial y^2}\label{eq:y_diff}
\end{align}

Aftering substitue Eq.\ref{eq:t_diff}, Eq.~\ref{eq:x_diff}, and Eq.~\ref{eq:y_diff} into Eq.~\ref{eq:p_diff}, $p(x,y,t)$, the probability of particle at $(x, y)$ at time $t$, satisfies

\begin{equation}\label{eq:lrws_heat}
  \frac{\partial \p}{\partial t} = \frac{(\triangle
    l)^2}{4 \delta} (\frac{\partial ^2 \p}{\partial x^2} +
  \frac{\partial^2 \p}{\partial y^2})
\end{equation}

where $D = \frac{(\triangle l)^2}{4 \delta}$ is the diffusion
coefficent. This above derivation shows a tight relationship between
the $2-$ dimensional discrete lattice random walks and the heat
equation.


\paragraph{Pearson's Random Walks (PRWs)}

Based on Pearson's problem and Rayleigh's answer,
Stadje \cite{stadje1987exact} and Masoliver et
al. \cite{masoliver1993some} considered a two-dimensional
continuous-time and continuous-space random walk, defined as Pearson's
random walks (PRWs) in this thesis. In PRWs, particle moves with
constant speed and with random directions distributed uniformly in
$[0, 2\pi)$. Moreover, the lengths of the straight-line paths and the
turn angles are stochastically independent. If the mean step length
approaches zero and the walking time is big enough, the behaviours of
particles in PRWs weakly converges to the Wiener
Process \cite{stadje1987exact}, which satisfies the traditional heat
equation. 
