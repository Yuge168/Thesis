
\section{Heat Content in Annulus}

Since the heat equation, also known as the diffusion equation, defined
in an annulus possesses explicit solutions, the analytical expressions
of heat content, derived by the integration over the whole domain, are
then accessible. In this section, the preliminary step is to solve the
initial-boundary value problem (IBVP) defined in the annulus. As the
solution of IBVP and its related quantities are in the form of the
infinite series, the numerical method will be used to approximate them
to validate the research methodology in the next chapter.


\subsection{\highlight[id=Yuge, comment={The writing of this subsection shared with Dave to revise.}]{Analytical Results}}\label{analytical results}

The $2$-dimensional heat equation \cite{crank1979mathematics}
Eq.~\ref{eq:polar_coordinate_diffusion} defined in the polar
coordinate system describes the heat distribution or temperature
varying in time and positions in the domain $\Omega$ shown in
Figure~\ref{fig:annulus}. $u(r, \theta, t)$ is the unknown function to
be solved, where $r$ is radial coordinate, $\theta$ is the angular
coordinate, and $t$ is the time. $D$ is the diffusion coefficient and
determines how fast $u$ changes in time.

\begin{equation}\label{eq:polar_coordinate_diffusion}
  u_t = D(u_{rr} + \frac{1}{r} u_r + \frac{1}{r^2} u_{\theta\theta})
\end{equation}

\begin{equation}\label{eq:Dirichlet_bc}
  u = 0 \; \; \; on \; \partial \Omega_1
\end{equation}

\begin{equation}\label{eq:Neumann_bc}
  u' = 0 \; \; \; on \; \partial \Omega_2
\end{equation}


\begin{equation}\label{eq:initial_bc}
  u(r, \theta, 0) = \frac{1}{|\Omega|}
\end{equation}

From the microscopic and probabilistic perspective, $u(r, \theta, t)$
is a probability density function, which gives the value of heat
particles at $(r,\theta)$ at time $t$. Eq.~\ref{eq:Dirichlet_bc} and
Eq.~\ref{eq:Neumann_bc} are the homogenous Dirichlet boundary
condition and the homogenous Neumann boundary condition,
respectively. In other words, the inner boundary $\partial \Omega_{1}$
is cooled to the zero temperature because the heat particles will be
absorbed when they encounter $\partial \Omega_1$. The outer boundary
$\partial \Omega_{2}$ is perfectly insulated since the heat particles
will be reflecting when they reach $\partial
\Omega_2$. Eq.~\ref{eq:initial_bc} states that the heat particles
distribute uniformly over the whole domain at time $t=0$, where
$|\Omega|$ equals the total area of the annulus.


\begin{figure}
  \centering
  \includegraphics[width=0.5\textwidth]{annulus.png}
  \caption{Assume the annulus $\Omega$ is a homogeneous and isotropic
    medium and has the inner boundary $\partial \Omega_1$ with radius
    $a$ and outer boundary $\partial \Omega_2$ with radius
    $b$. \label{fig:annulus}}
\end{figure}



\subsubsection{Solving Heat Equation}

Generally, before solving the heat equation, it is convenient and
efficient to generate a group of dimensionless variables by
dimensional analysis. The benefit of dimensional analysis is that many
physical parameters can be combined into a smaller number of unitless
variables, which do not depend on the unit of the measurements and can
also describe the phenomenon or system of interest
\cite{barenblatt1996scaling}.


Let $\mu = b/a$ be the dimensionless radius ratio, $\tau =
\frac{Dt}{a^2}$ be the dimensionless time, and $\hat r = \frac{r}{a}$
be the unitless radius. Substitute these dimensionless variables into
Eq.~\ref{eq:polar_coordinate_diffusion} and rewrite it as

\begin{equation}\label{eq:DA_polar_diffusion}
  u_\tau = (u_{\hat r \hat r} + \frac{1}{\hat r} u_{\hat r} + \frac{1}{\hat r ^2} u_{\theta\theta})
\end{equation}

With the uniform initial condition,
\begin{equation}\label{eq:DA_initial_bc}
  u(\hat r, \theta, 0) = \frac{1}{|\Omega|}
\end{equation}

With the homogenous boundary conditions
\begin{equation}\label{eq:DA_Dirichlet_bc}
  u(1, \theta, \tau) = 0
\end{equation}

\begin{equation}\label{eq:DA_Neumann_bc}
  u'(\mu, \theta, \tau) = 0
\end{equation}



After implementing the separation of variables method
\cite{crank1979mathematics}, the solutions of
Eq.~\ref{eq:DA_polar_diffusion} with conditions
Eq.~\ref{eq:DA_initial_bc}, Eq.~\ref{eq:DA_Dirichlet_bc}, and
Eq.~\ref{eq:DA_Neumann_bc} are

\begin{equation}\label{eq:annulus_solutions_u}
  u(\hat r, \theta, \tau) = \sum_{n=1}^{\infty}
  \tilde{c_{0,n}} \bigg\{J_0(\sqrt{\lambda_{0,n}})
  Y_0(\sqrt{\lambda_{0,n}} \hat r) -
  Y_0(\sqrt{\lambda_{0,n}}) J_0(\sqrt{\lambda_{0,n}} \hat
  r)\bigg\} e^{-\lambda_{0,n}\tau}
\end{equation}

where

\begin{equation}\label{eq:coeff_u}
  \tilde{c_{0,n}} = \frac{1}{(\mu^2 - 1)}
\frac{1}{\bigg[\frac{J_0(\sqrt{\lambda_{0,n}})}{J'_0(\mu
      \sqrt{\lambda_{0,n}})}\bigg]^2 -1}
\end{equation}


Eigenvalues $\lambda_{0, n}$ $(n \in \mathbb{N}_{+})$ appeared in
Eq.~\ref{eq:annulus_solutions_u} and Eq.~\ref{eq:coeff_u} is the $n$th
positive root of the eigenfunction Eq.~\ref{eq:eigenfunction}, which
is a cross-product of the Bessel function \cite{watson1995treatise}

\begin{equation}\label{eq:eigenfunction}
  F_0(\lambda) = J_0(\sqrt{\lambda}) Y_0'(\sqrt{\lambda} \mu) -
  J_0'(\sqrt{\lambda} \mu) Y_0(\sqrt{\lambda})
\end{equation}


\subsubsection{Heat Content (Survival Probability)}

The amount of heat contained in $\Omega$ at the moment $\tau > 0$
defined as heat content $Q_{\Omega}(\tau)$, which is an alternative
terminology of survival probability $S(\tau)$ in some mathematical
literatures \cite{birkhoff1954note} \cite{van1994heat}
\cite{gilkey1994heat}. $S(\tau)$ is proportional to $Q_{\Omega}(\tau)$
\cite{kalinay2011survival}, which gives the probability of the
particles remain diffusing in the domain $\Omega$ at time $\tau > 0$
\cite{aalen2008survival}. Survival probability can be expressed by

\begin{equation}\label{eq:annulus_analytical_s}
  \begin{split}
    S(\tau) &= \int_{0}^{2\pi} d\theta \int_{1}^{\mu} \hat r d \hat r
    u(\hat r, \theta, \tau)\\ &= \sum_{n=1}^{\infty} \frac{4}{\mu^2 -
      1} \frac{1}{\lambda_{0,n}
      \bigg\{\bigg[\frac{J_0(\sqrt{\lambda_{0,n}})}{J'_0(\mu
          \sqrt{\lambda_{0,n}})}\bigg]^2 -1\bigg\}} e^{-\lambda_{0, n}
      \tau}
  \end{split}
\end{equation}

Eq.~\ref{eq:annulus_analytical_s} reveals some basic properties of
$S(\tau)$. Firstly, when $\tau=0$, the survival probability is $1$
since all the particles are just generated over the whole domain and
not be absorbed by $\Omega_1$. Secondly, $S(\tau)$ is a convergent
series with multiexponential decay. Thirdly, $S(\tau)$ interconnects
the overall geometric characteristics of $\Omega$. For example, the
decay rate of $S(\tau)$ in a short time heavily depends on the
geometrical features of $\Omega_1$, as only the particles inserted
close to $\Omega_1$ have the high probabilities of being
absorbed. Finally, as the lone-time limit, $S(\tau)$ is represented by
the lowest eigenvalue $\lambda_{0,1}$.


\subsubsection{Mean First-Passage Time}

The first passage phenomena play a fundamental role in stochastic
processes triggered by a first-passage event
\cite{van1992stochastic}. In this thesis, we only focus on the
stochastic evolution of $u(\hat r, \theta, \tau)$ until the
first-passage time, at which a heat particle reaches any sites of
target boundary $\Omega_1$ for the first time. Similarly, another
essential first-passage-related quantity is the first-passage
probability, which is a probability of the diffusing heat particles
hitting a specified site or a set of sites at a specified time for the
first time \cite{redner2001guide}. All the first-passage
characteristics can be expressed in terms of the first-passage
probability. For example, the survival probability of heat particles
at time $\tau$ calculated in the last subsection is

\begin{equation}\label{eq:pdf_fpt}
  f(\tau) = - \frac{\partial S(\tau)}{\partial \tau}
\end{equation}

where $f(\tau)$ is the first-passage probability to the target
boundary $\Omega_1$ at time $\tau$ regardless of particles' stop
positions. By the definition, the $n$th moment of the exit time
\cite{redner2001guide} is

\begin{equation}\label{eq:nth_fpt}
  \begin{split}
    \langle \tau^n \rangle &= \int_{0}^{\infty} \tau^n f(\tau) d\tau \\
    &= - \int_{0}^{\infty} \tau^n  \frac{\partial S(\tau)}{\partial \tau} d\tau \\
    &= -\tau^n S(\tau) |_{0}^{\infty} + n\int_{0}^{\infty} \tau^{n-1}S(\tau) d\tau
\end{split}
\end{equation}


Substitue $n=1$ in Eq.~\ref{eq:nth_fpt}, the mean-first passage time
$\langle \tau \rangle$, also called the average first-passage time, of
heat particles implies an overall property of the system and can be
expressed as

\begin{equation}\label{eq:mfpt}
  \begin{split}
    \langle \tau \rangle &= \int_{0}^{\infty} \tau dS\\
    &=\sum_{n=1}^{\infty} \frac{4}{\mu^2 - 1}
    \frac{1}{\lambda^2_{0,n}\bigg\{\bigg[\frac{J_0(\sqrt{\lambda_{0,n}})}{J'_0(\mu\sqrt{\lambda_{0,n}})}\bigg]^2
      -1\bigg\}}
  \end{split}
\end{equation}

