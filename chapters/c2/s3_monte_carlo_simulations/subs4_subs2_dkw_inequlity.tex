\subsubsection{Dvoretzky–Kiefer–Wolfowitz (DKW) inequality}

Although the sample size estimated by the general method does not
depend on the geometries characterized by the random walk models, how
long the simulation will run is unknown, and the final calculated
sample size will be larger than the really necessary value
sometimes. Therefore, an alternative method, named the
Dvoretzky–Kiefer–Wolfowitz (DKW) inequality
\cite{dvoretzky1956asymptotic}, is proposed for the sample size
determination without simulating random walk models and considering
the shape of objects in the images.

Let $F_N(x)$ denote the empirical distribution functions (empirical
CDF) for a sample of $N$ real-valued $i.i.d.$ random variables,
$X_{1}, ... , X_{N}$, with continuous cumulative distrbution function
(CDF) $F(x)$. The DKW inequality, as expressed in Eq.$(2.18)$, bounds the
probability that the random function $F_{N}(x)$ differs from the true
$F(x)$ by more than a given constant $\varepsilon$
\cite{dvoretzky1956asymptotic}.

\begin{equation}
  Pr(\sup_{x \in \mathbb{R}} |F_{N}(x) - F(x)| > \varepsilon) \leq
  2e^{-2N\varepsilon^2} \;\; \;\; for \; every \; \varepsilon > 0
\end{equation}

The equally spaced confidence bounds or simultaneous band around the
$F_{N}$ encompassing the entire $F(x)$ can be expressed by
\begin{equation}
  F_{N}(x) - \varepsilon \leq F(x) \leq F_N(x) + \varepsilon \; \; \; \; 
\end{equation}

On the other hand, assume the simultaneous band produced by Eq.$(2.19)$
containing the $F(x)$ at a given confidence level $1-\alpha$, the
interval $\varepsilon$ can be calculated by

\begin{equation}
  \varepsilon = \sqrt{\frac{\ln{\frac{2}{\alpha}}}{2N}}
\end{equation}


Given a converge probability $\alpha$ and constant $\varepsilon$, it is
straightforward to estimate the sample size $N$ in the fixed-time step
Monte Carlo simulations in any images by Eq.$(2.20)$.

