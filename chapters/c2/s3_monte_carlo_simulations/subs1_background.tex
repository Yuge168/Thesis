\section{Monte Carlo Simulations}

In this section, several generally utilized numerical methods
\cite{grossmann2007numerical}\cite{zlamal1968finite}
\cite{eymard2000finite} \cite{attaway1991boundary} for solving the
heat equation, and their limitations in practice are presented. Also,
one of the non-deterministic algorithms, Monte Carlo methods (MCM)
\cite{rubinstein2016simulation} \cite{kroese2014monte}, and its
application in approximating the solution of the PDEs are proposed. As
the weaknesses and challenges of applying the numerical techniques in
solving $2-$dimensional heat equation defined in the real root images
with millions of pixels and extremely complex root systems, the
alternative fixed-time step Monte Carlo simulations, lattice random
walks (LRWs) and Pearson's random walks (PRWs), are designed. The most
outstanding advantage of the proposed random walk models is that the
integration, named the heat content, can be approximated directly
based upon the probabilistic interpretation of Brownian motion and the
heat equation.  Finally, the methods to analyze the output of the
Monte Carlo simulations and solve the sampling-related problems in the
simulations are brought up theoretically.



\subsection{Background}\label{background}

In the subsection~\ref{analytical results}, the analytical solutions
of the heat equation defined in the annulus with the initial and
boundary conditions have been derived. The analytical heat content,
calculated by the integration over $\Omega$, implies the geometrical
properties of the annulus. However, the analytical method for solving
the heat equation has many restrictions, and its applications to
practical problems will exhibit difficulties. Firstly, the numerical
evaluation of the analytical solutions is usually by no means trivial
because they are in the form of infinite series. Secondly, either
irregular geometries or discontinuities lead to the complexities, so
the explicit algebraic solutions are close to non-existed. Thirdly,
the purely analytical techniques can apply strictly only to the linear
form of the boundary conditions and to constant diffusion properties
\cite{crank1979mathematics}. Therefore, numerical methods and computer
simulations are more helpful and applicable to find solutions to the
heat equation than calculating pure analytical solutions.
