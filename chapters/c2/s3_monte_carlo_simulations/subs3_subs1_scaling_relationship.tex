\subsubsection{Relationship between $t$ and $\tau$}

Particles' average one-step displacement $\triangle l$ in the
fixed-time step Monte Carlo simulations, LRWs and PRWs, are associated
with the time step is $\delta$:

\begin{equation}
  \triangle l = 2 \sqrt{D \delta}
\end{equation}
where $D$ is the diffusion coefficent.

Eq.$(2.15)$ implies that the time step $\delta$ must be designed small
enough to make sure that $\triangle l$ is shorter than the smallest
geometrical features of the boundaries. Thus, $\triangle l$ should
equal or be less than one-pixel size in the simulations. Furthermore,
the $\delta$ is regarded as a fundamental bridge between particles'
number of steps $t$ and unitless continuous-time $\tau$,

\begin{equation}
 \tau = t \delta = \frac{(\triangle l)^2 t}{4D}
\end{equation}
where $D$ is 1.

When running the LRWs in the annulus, $\triangle l$ is always
$\frac{1}{100}$ since particle's step length is as same as one-pixel
size. However, $\triangle l$ is related to the particle's step length
in PRWs. If particle's step length is $0.5$, a half of a pixel, then
$\triangle l$ equals $\frac{1}{100} \times \frac{1}{2} =
\frac{1}{200}$.  Similiarly, when the step length in PRWs is $0.1$,
then $\triangle l$ is $\frac{1}{1000}$. 
