\subsubsection{Probabilistic Interpretation}\label{probabilistic interpretation}

In the subsection~\ref{analytical results}, the heat equation
describes the temperature distribution of a homogeneous and isotropic
domain \cite{varadhan1980lectures}, and its solution characterizes how
the temperature changes over the position and time. From the
probabilistic perspective, the heat equation and its solution can also
be understood by the Brownian
motion \cite{brown1828microscopical}. The Brownian motion also called
the Wiener process, is a continuous-time and continuous-space
stochastic process \cite{karlin2014first} with the continuous sample
paths and stationary independent
increments \cite{ito2012diffusion}. This process also has the Markov
property: the future state depends only on the present
state \cite{bharucha2012elements}. In the probability theory, if a
large number of free particles undergoing the Brownian motion
independently, the density of particles at a specific time becomes a
deterministic process called diffusion, which satisfies the heat
equation \cite{kac1947random}\cite{varadhan1980lectures}.



\paragraph{Survival Probability}


\newcommand{\cpu}{\rho(\hat r, \theta, \tau | \hat{r_0}, \theta_0, 0)}
\newcommand{\cpuf}[1]{\ensuremath{\rho_{#1}}} 
\newcommand{\cpus}[2]{\ensuremath{\rho_{#1#2}}}
% Define the conditional probability u, which will be used frequently. 
                                 

For simplicity, we only investigate the probabilistic interpretation
of the heat equation defined in the annulus with the boundary and
initial conditions as same as described in the
subsection~\ref{analytical results}. Consider a particle undergoing
the Brownian motion from $(\hat{r_0}, \theta_0) \in \Omega$ at
$\tau=0$, and let $\cpu$ be the conditional probability of finding the
particle at $(\hat r, \theta) \in \Omega$ at time $\tau>0$. Moreover,
particle's initial position $(\hat{r_0}, \theta_0)$ is distributed
uniformly over the whole domain $\Omega$. $\cpu$ satisfies the following
equations

\begin{align}
  \cpuf{\tau} & = \cpus{\hat r}{\hat r} + \frac{1}{\hat r} \cpuf{\hat r} + \frac{1}{\hat{r}^2} \cpus{\theta}{\theta}
  \qquad\text{for $(\hat r, \theta) \in \Omega$}\label{eq:polar_coordinate_diffusion_conditional_prob} \\
  \rho & = 0
  \qquad\text{for $(\hat r, \theta) \in \partial \Omega_1$}\label{eq:Dirichlet_bc_conditional_prob} \\
  \cpuf{\hat r} & = 0
  \qquad\text{for $(\hat r, \theta) \in \partial \Omega_2$}\label{eq:Neumann_bc_conditional_prob} \\
\end{align}


The "local" survival probability $S(\tau | \hat{r_0}, \theta_0, 0)$,
represents the probability that a particle, localized at
$(\hat{r_0}, \theta_0)$ at $\tau=0$, keeps diffusing in $\Omega$ at
time $\tau > 0$ and is unabsorbed by the boundary $\Omega_1$. 

\begin{equation}\label{eq:cpu_s_local}
  S(\tau | \hat{r_0}, \theta_0, 0) = \iint_{\Omega} \hat r \cpu d \hat r d\theta 
\end{equation}

Our interest is the "global" survival probability $S(\tau)$, the average of the local
survival probability over $\Omega$, expressed as

\begin{equation}\label{eq:cpu_s_global}
  S(\tau) = \frac{1}{|\Omega|}\iint_{\Omega} \hat{r_0} S(\tau | \hat{r_0}, \theta_0, 0) d \hat{r_0} d \theta_0
\end{equation}

Eq.~\ref{eq:cpu_s_local} and Eq.~\ref{eq:cpu_s_global} reveal that the
heat content $Q_{\Omega}(\tau)$ expressed in
Eq.~\ref{eq:annulus_analytical_s} is proportional to the survival
probability $S(\tau)$ \cite{kalinay2011survival}. 


\paragraph{Mean First-Passage Time}

The first passage phenomena play a fundamental role in the stochastic
processes triggered by a first-passage
event \cite{van1992stochastic}. In this thesis, one of the essential
first-passage-related quantities is the first-passage time or the
first-hitting time \cite{redner2001guide}, which is the time taken by
particle undergoing the Brownian motion from an initial position to
any sites of $\Omega_1$ for the first time. Particles' mean
first-passage time $\langle \tau \rangle$, also called the average
first-passage time, has a closed relationship with the survival
probability \cite{redner2001guide}

\begin{equation}\label{eq:mfpt_conditional_prob}
  \begin{split}
    \langle \tau \rangle &= \int_{0}^{\infty} \tau dS(\tau)\\
    &=\sum_{n=1}^{\infty} \frac{4}{\mu^2 - 1}
    \frac{1}{\lambda^2_{0,n}\bigg\{\bigg[\frac{J_0(\sqrt{\lambda_{0,n}})}{J'_0(\mu\sqrt{\lambda_{0,n}})}\bigg]^2
      -1\bigg\}}
  \end{split}
\end{equation}

From Eq.~\ref{eq:mfpt_conditional_prob} and
Eq.~\ref{eq:eigenfunction}, it is clear that $\langle \tau \rangle$
implies an overall property of the annulus since it only depends on
the radius ratio of annulus $\mu$.




\paragraph{Brownian Motion and Random Walks}

Brownian motion, the irregular motion of individual particles, has
been existed for a long time before the random-walk theory was
developed. At the beginning of the twentieth century, the term, random
walk, was initially proposed by Karl
Pearson \cite{pearson1905problem}. He utilized the isotropic planar
random flights to model how mosquitoes migrate and invade randomly in
the cleared jungle regions. At each time step, the mosquito moves to a
random direction with a fixed step
length. Rayleigh \cite{rayleigh1905problem} answered Pearson's
question in the same year, the distributions of mosquitos after many
steps have been taken, is identical to superposition the sound
vibrations with unit amplitude and arbitrary
phase \cite{de2012flying}. At almost the same time, Louis
Bachelier \cite{bachelier1900theorie} designed a model for the
financial time series by the random walks. He also explored the
connection between discrete random walks and the continuous heat
equation. During the development of random-walk theory, many other
scientific fields, including the random processes, random noise,
spectral analysis, and stochastic equations, were developed by some
physicists \cite{einstein1905electrodynamics} \cite{einstein1906theory} \cite{smoluchowski1916drei}. The
continuous Brownian motion can be considered as the limit of the
discrete random walks as the time and space increments go to
zero \cite{lawler2010random}\cite{varadhan1980lectures}.


\paragraph{Lattice Random Walks (LRWs)}

Let us consider a large number of particles performing the simple
random walk on the $d$-dimensional integer grid $\mathbb{Z}^d$. It is
a discrete-space and discrete-time symmetric hopping
process \cite{redner2001guide} on the lattice. At each time step, the
particle moves to one of its $2d$ nearest neighbours with probability
$\frac{1}{2d}$. If $d \leq 2$, the random walk is
recurrent \cite{hughes1998random}, which means the particle will
return to its origin infinitely often with the probability $1$. If
$d \geq 3$, the random walk is transient \cite{hughes1998random},
which indicates the particle will return to its origin only finitely
often with the probability $1$ \cite{hughes1998random}.

$2$-dimensional lattice random walks
(LRWs) are considered and bec.

Let $\triangle l$ be the distance between
two sites and $\delta$ be the time step. Define $p(x,t| x_{S})$ as the
conditional probability of a particle to be in position $x$ at time
$t$ starting from $x_{S}$ at time $t=0$. $p(x, t + \delta |x_{S})$ is
defined as probability of a particle to be in position $x$ at time $t
+ \delta$ starting from $x_{S}$ at time $t=0$, \cite{lawler2010random}


\begin{equation}\label{eq:lrws_next}
  p(x, t + \delta | x_{S}) = \frac{p(x - \triangle l e_{x}, t| x_{S}) +
    p(x + \triangle l e_{x}, t| x_{S}) + p(x - \triangle l e_{y}, t|
    x_{S}) + p(x + \triangle l e_{y}, t| x_{S})}{4}
\end{equation}

where $e_x$ and $e_y$ are unit vectors of $x-$ axis and $y-$ axis, respectively.


When $\delta \rightarrow 0$, $\triangle l \rightarrow 0$, and $\delta \sim (\triangle l)^2$, 

\begin{equation}\label{eq:lrws_prob}
  p(x, t | x_{S}) + \delta p(x, t | x_{S}) = p(x, t | x_{S}) +
  \frac{(\triangle l)^2}{4} (\frac{\partial ^2 p(x, t|
    x_{S})}{\partial x^2} + \frac{\partial^2 p (x, t| x_{S})}{\partial
    y^2})
\end{equation}

Finally, the $2$-dimensional heat equation is

\begin{equation}\label{eq:lrws_heat}
  \frac{\partial p(x, t| x_{S})}{\partial t} = \frac{(\triangle
    l)^2}{4 \delta} (\frac{\partial ^2 p(x, t| x_{S})}{\partial x^2} +
  \frac{\partial^2 p (x, t| x_{S})}{\partial y^2})
\end{equation}
where $D = \frac{(\triangle l)^2}{4 \delta}$ is the diffusion coefficent. This derivation shows a tight relationship between lattice random walks and diffusion.



\paragraph{Pearson's Random Walks (PRWs)}

Based on Pearson's problem and Rayleigh's answer, Stadje
\cite{stadje1987exact} and Masoliver et al. \cite{masoliver1993some} considered a two-dimensional continuous-time and continuous-space random walk, defined as Pearson's random walks (PRWs) in this thesis. In PRWs, particle moves with constant speed and with random directions distributed uniformly in $[0, 2\pi)$. Moreover, the lengths of the straight-line paths and the turn angles are stochastically independent. If the mean step length approaches zero and the walking time is big enough, the behaviours of particles in PRWs weakly converges to the Wiener Process \cite{stadje1987exact}, which satisfies the traditional heat equation.
