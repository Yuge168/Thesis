\subsubsection{Kaplan-Meier Estimator}

The general definition of the survival time is the time starting from
a specified point to the occurrence of a given
event \cite{bewick2004statistics}, such as death, pregnancy, job loss,
etc. Also, the analysis of the group of survival data is called
survival analysis \cite{altman1990practical}. In the survival
analysis, three kinds of situations will affect the subjects' survival
time \cite{goel2010understanding}. Firstly, the subjects are
uncooperative and refused to continue to participate in the
research. Secondly, some subjects do not experience the event before
the end of the study, but they would have experienced the event if
they keep being observed. Finally, the researchers lose touch with the
subjects in the middle of the investigation. In practice, since these
subjects have partial information about survival, the scientists will
label the above circumstances as censored
observations \cite{bewick2004statistics} instead of ignoring them and
decreasing sample size.

In clinical trials or community trials, Kaplan-Meier
Estimator \cite{kaplan1958nonparametric}, a non-parametric analysis,
is a commonly applied statistical method in the survival analysis for
the measurement of the fraction of the survival time after the
treatment \cite{aalen2008survival} and for generating the
corresponding survival curve. It also works well with the mentioned
three difficult situations. With various
assumptions \cite{etikan2017kaplan} \cite{goel2010understanding}, the
Kaplan-Meier survival curve can be created and provides the
probability of surviving in a given length of time while considering
the time in many small intervals \cite{altman1990practical}.


Let $0 < t_1 < t_2 < ...$ be the distinct increasing observed times,
or the number of steps taken by the particle hitting the
absorbing boundary, in the sample. Let $n_i$ be the number of particles
who either have not yet stopped moving up to time $t_i$ or else who
are absorbed on the target boundary at time $t_i$ in the
simulations. Let $d_i$ the number of particles hitting the target
boundary at time $t_i$. The Kaplan-Meier or product-limit estimator
$\widehat{S}(t)$ of the survival function of the numerical simulation
$S(t)$ is \cite{aalen2008survival}

\begin{equation}
  \widehat{S}(t) = \prod_{i:t_i \leq t} (1 - \frac{d_i}{n_i})
\end{equation}




