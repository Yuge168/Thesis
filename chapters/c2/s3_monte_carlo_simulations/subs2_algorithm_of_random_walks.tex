\subsection{Algorithms of Random Walks}

In the subsection~\ref{background}, the practical challenges of solving the heat equation defined in the real root images with millions pixels by numerical methods and Monte Carlo simulations are revealed and the probabilistic interpretaion of the heat equation, survival probability, and random walks are introduced. From Eq.~\ref{eq:cpu_s_global} and Eq.~\ref{eq:mfpt}, it is easy to explore that the final goal in this thesis is the aproximation of the heat content, or the survival probability, defined as the intergration of the solution of the heat equation over the whole domain, which is only dependent on the time variable. Moreover, our interest is only related to the time when the first-passage event happens in the stochastic process, that is, the first-passage time. Therefore, instead of trying to approximate the solution of the heat equation and the integration of the solution, two algorithms of random walks are designed in this subsection to mimic particles' first-passage time by $2-$ dimensional fixed-time step Monte Carlo simulations in the real $2-$ dimensional root images for approximating the integration as expressed in Eq.~\ref{eq:cpu_s_global} and Eq.~\ref{eq:mfpt}.


\subsubsection{Lattice Random Walks}




\subsubsection{Pearson's Random Walks}
