
\subsection{\highlight[id=Yuge, comment={Dave, please have a look at this section and give me feedback. Thanks!}]{Background}}\label{background}

Instead of understanding the heat equation
Eq.~\ref{eq:Cartesian_heat_equation} and its solution from the
physical point of view at the macroscopic scale, it will be easier to
think about it from the microscopic peospective: a large amount of
randomly moving particles. 



\subsubsection{\highlight[id=Yuge]{Brownian Motion}}\label{Brownian_Motion}

In the early 19th century, the French scientist Jean Baptiste Joseph
Fourier presented the remarkable heat equation describing the heat
conduction in a solid and proposed a technique, named the separation
of variable, to solve the equation subject to the initial and boundary
conditions \cite{baron1878analytical}. In 1827, the Scottish botanist
Robert Brown discovered the microscopic motion of the pollen grains,
which suspended in the water going in an erratic and highly irregular
zigzag pattern \cite{brown1828microscopical}. Based on his report,
other scientists verified the existence of the Brownian motion,
defined as the irregular motion of individual particles, in other
fields. Originally, in 1905, Albert Einstein
\cite{einstein1905electrodynamics} developed a satisfactory
explanation of the Brownian motion from the jostling of the particle
by the molecules of water. He also utilized a probabilistic model
showing that the Brownian motion, in a certain sense, provided a
solution to Fourier's heat equation. In 1918, Norbert Wiener began
considering the Brownian motion as a mathematical random process with
the required properties in a rigorous way
\cite{wiener1923differential}. Therefore, Brownian motion, also called
the Wiener process, is a continuous-time and continuous-space
stochastic process with the continuous sample paths and stationary
independent increments \cite{ito2012diffusion}
\cite{karlin2014first}. The stationary increment of the Brownian
motion is a statement of time homogeneity; that is, the distribution
of the particle's displacement in a time interval depends only on the
length of the time interval. Brownian particles' independent
increments imply the Markov property since the particle's displacement
is independent of its past displacement. Therefore, Brownian motion
can be considered as one of the fundamental Markov processes.


In the probability theory, if a large number of free particles
undergoing the Brownian motion independently, the density of particles
at a specific time becomes a deterministic process, called the
diffusion process, which satisfies the diffusion equation
\cite{kac1947random}\cite{varadhan1980lectures} or heat equation. Let
$P(\bm{s}, t | \bm{s_0})$ be the probability density for Brownian
motion at a point $\bm{r} \in \Omega$ at time $t>0$ started from
$\bm{s_0} \in \Omega$ at $t=0$ satisfying the following equations


\begin{align}
  \frac{\partial P(\bm{s}, t | \bm{s_0})}{\partial t} &= \Delta D P(\bm{s}, t | \bm{s_0}) \label{eq:brownian_motion_heat_eq} \\
  P(\bm{s}, t| \bm{s_0}) &= \delta(\bm{s} - \bm{s_0}) \qquad \text{for $t=0$} \label{eq:initial_brownian_motion} \\
  P(\bm{s}, t| \bm{s_0}) &= 0 \qquad \text{for $t>0$ and $\bm{s} \in \partial \Omega$} \label{eq:Dirichlet_bc_brownian_motion}
\end{align}


Note, $D$ is the diffusion constant. To ensure the existence and
uniqueness of the solution, the heat equation
Eq.~\ref{eq:brownian_motion_heat_eq} has to be supplemented by the
initial condition Eq.~\ref{eq:initial_brownian_motion} with the Dirac
$\delta -$ distribution and the Dirichlet boundary condition
Eq.~\ref{eq:Dirichlet_bc_brownian_motion}, which indicates the
absorption and disappearance of the particle hitting the boundary
$\partial \Omega$.


\subsubsection{\highlight[id=Yuge]{Random-Walk Theory}}


Brownian motion has existed for a long time before the considerable
development of the random-walk theory. At the beginning of the
twentieth century, the term, random walk, was initially proposed by
Karl Pearson \cite{pearson1905problem}. He utilized the isotropic
planar random flights to model the erratic migration and invade of
mosquitoes in the cleared jungle regions. In his model, the mosquito
moves to a random direction with a fixed step length at each time
step. Moreover, he asked a question in a paper about the distribution
of mosquitos after many steps have been taken. Actually, earlier in
1880, John William Strutt and Baron Rayleigh have used the similar
concept of random walk in analyzing a certain random vibration problem
\cite{rayleigh1880xii}. Therefore, in 1905, Rayleigh
\cite{rayleigh1905problem} answered Pearson's question, that is, the
distribution is identical to superposition the sound vibrations with
unit amplitude and arbitrary phase \cite{de2012flying}. At almost the
same time, Louis Bachelier designed a model for the financial time
series based on the random walks. Louis Bachelier also explored the
relationship between discrete random walks and the continuous heat
equation \cite{bachelier1900theorie}. While Rayleigh only studied the
symmetric random walk, the nonsymmetric case was considered by Mark
Kac in 1947 \cite{kac1947random}. Kac provided a proof that the
nonsymmetric random walk is the mathematical analogy to the
Smoluchowski's diffusion with drift
\cite{smoluchowski1916drei}. Moreover, during the development of
random-walk theory, many other scientific fields, including the random
processes, random noise, spectral analysis, and stochastic equations,
were developed by some physicists \cite{einstein1905electrodynamics}
\cite{einstein1906theory} \cite{smoluchowski1916drei}.


\subsubsection{\highlight[id=Yuge]{Lattice Random Walks (LRWs)}}

The continuous Brownian motion is the scaling limit of the discrete
random walks as the time and space increments approach zero
\cite{lawler2010random}\cite{varadhan1980lectures}. Let us consider a
particle performing the simple random walk on the $d$-dimensional
integer grid $\mathbb{Z}^d$. It is a discrete-space and discrete-time
symmetric hopping process \cite{redner2001guide} on the lattice. At
each time step, the particle moves to one of its $2d$ nearest
neighbours with probability $\frac{1}{2d}$. If $d \leq 2$, the random
walk is recurrent, which means the particle will return to its origin
infinitely often with the probability $1$. If $d \geq 3$, the random
walk is transient, which denotes the particle will return to its
origin only finitely often with the probability $1$
\cite{hughes1998random} \cite{lawler2010random}.


From the probalistic view, only $2$-dimensional lattice random walks
(LRWs) and its connection with the heat equation are introduced here
since this thesis aims to explore and characterize the shape of roots
in the $2-$dimensional images. In the random walk, let $\triangle l$
be the distance between two sites in the lattice and $\delta$ be the
time between two steps. Let $P(x, y, t + \delta | \bm{s_0})$ be the
conditional probability of a particle to be in position $(x, y)$ at
time $t + \delta$ starting from $\bm{s_0}$ at time $t=0$. Without
hitting the boundary $\partial \Omega$ of the domain $\Omega$, $P(x,
y, t + \delta | \bm{s_0})$ can be expressed by the stochastic
difference equation

\begin{equation}\label{eq:master_eq}
  P(x, y, t + \delta | \bm{s_0}) = \frac{P(x - \triangle l, y, t | \bm{s_0}) +  P(x + \triangle l, y, t | \bm{s_0}) + P(x, y - \triangle l, t | \bm{s_0}) + P(x, y + \triangle l, t | \bm{s_0})}{4}
\end{equation}

After rewriting the master Eq.~\ref{eq:master_eq} by subtracting $P(x, y, t  | \bm{s})$ and dividing by $\delta$,

\begin{equation}
  \begin{split}
  \frac{P(x, y, t + \delta| \bm{s_0}) - P(x, y, t | \bm{s_0})}{\delta} &= \frac{1}{4\delta} \Big( P(x - \triangle l, y, t | \bm{s_0}) + P(x + \triangle l, y, t | \bm{s_0}) \label{eq:diff_master_eq} \\
  &\quad + P(x, y - \triangle l, t  | \bm{s_0}) + P(x, y + \triangle l, t | \bm{s_0}) - 4 P(x, y, t| \bm{s_0}) \Big) 
  \end{split}
\end{equation}


As $\delta \rightarrow 0$, the left side of
Eq.~\ref{eq:diff_master_eq} is the definition of a derivative repect
to time

\begin{equation}\label{eq:time_derivative}
  P(x, y, t + \delta | \bm{s_0}) \approx P(x, y, t | \bm{s_0}) +
  \frac{\partial P(x, y, t | \bm{s_0})}{\partial t}\delta
\end{equation}


For the right side of Eq.~\ref{eq:diff_master_eq}, expanding the
conditional probability $P$ as Taylor series and ignoring the higher
order terms, there has

\begin{align}
  P(x \pm \triangle l, y, t | \bm{s_0}) \approx P(x, y, t | \bm{s_0}) \pm \frac{\partial P(x, y, t | \bm{s_0})}{\partial x} \triangle l + \frac{1}{2} \frac{\partial ^2 P(x, y, t | \bm{s_0})}{\partial x^2}(\triangle l)^2 \quad \label{eq:Taylor_x}\\
  P(x, y \pm \triangle l, t | \bm{s_0}) \approx P(x, y, t | \bm{s_0}) \pm \frac{\partial P(x, y, t | \bm{s_0})}{\partial y} \triangle l + \frac{1}{2} \frac{\partial ^2 P(x, y, t | \bm{s_0})}{\partial y^2}(\triangle l)^2 \quad \label{eq:Taylor_y}
\end{align}


Substituting Eq.~\ref{eq:Taylor_x} and Eq.~\ref{eq:Taylor_y} into
right side of Eq.~\ref{eq:diff_master_eq} and implementing
all the obvious cancellations, the Eq.~\ref{eq:diff_master_eq} is
rearranged as

\begin{equation}\label{eq:lrw_diffusion_eq}
  \frac{\partial P(x, y, t | \bm{s_0})}{\partial t} \approx
  \frac{(\triangle l)^2}{4 \delta} \Big( \frac{\partial ^2 P(x, y, t |
    \bm{s_0})}{\partial x^2} + \frac{\partial ^2 P(x, y, t |
    \bm{s_0})}{\partial y^2}\Big)
\end{equation}

When $\tau \rightarrow 0$, $\triangle l \rightarrow 0$, and $\tau \sim (\triangle l)^2$, Eq.~\ref{eq:lrw_diffusion_eq} becomes an exact
relation.

\begin{equation}\label{eq:limit_lrw_diffusion_eq}
  \frac{\partial P(x, y, t | \bm{s_0})}{\partial t} = D \Big( \frac{\partial ^2 P(x, y, t | \bm{s_0})}{\partial x^2} + \frac{\partial ^2 P(x, y, t |\bm{s_0})}{\partial y^2}\Big)
\end{equation}

where $D = \frac{(\triangle l)^2}{4 \delta}$ is the diffusion
coefficent. This above derivation shows a relationship between a very
simple microscopic discrete random walk model and a well-known
macroscopic probalistic partial differential equation.





    











