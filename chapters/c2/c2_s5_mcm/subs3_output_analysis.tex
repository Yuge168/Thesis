



\begin{comment}
  \subsection{Output Analysis}

Particles' first-passage time $t$, also called the first-hitting time,
is the output of the lattice random walks (LRWs). More specifically,
$t$ is the number of steps taken by the particle encountering any
positions of the target boundary for the first time. Since the
first-hitting-time model is a sub-class of survival analysis in
statistics \cite{altman1990practical}, it is straightforward to
estimate the survival function $S(t)$ of the numerical simulation by
the Kaplan-Meier estimator $\widehat{S}(t)$
\cite{kleinbaum2005competing}. $S(t)$ provides the probability that
the particle remains simulating in the LRWs beyond any specified
time. Moreover, the pointwise upper and lower confidence interval of
$S(t)$ can be estimated by the Greenwood's exponential formula
\cite{hosmer2011applied}. In this subsection, the Kaplan-Meier
estimator and pointwise confidence intervals are presented
theoretically. However, in practice, the existing Python module,
lifeline \cite{davidson2019lifelines}, helps implement the
estimation. Based on the scaling relationship between the number of
steps $t$ and the unitless time $\tau$, non-parametric two-sample
statistical tests are used to compare $S(t)$ and $S(\tau)$ for
validating the research methodology.



\subsubsection{Kaplan-Meier Estimator}


The survival time is defined as the time starting from a specified
point to the occurrence of a given event \cite{bewick2004statistics},
such as death, pregnancy, job loss, etc. The analysis of the group of
survival time data is called survival analysis
\cite{altman1990practical}. In the survival analysis, three kinds of
situations will affect the subjects' survival time
\cite{goel2010understanding}. Firstly, the subjects are uncooperative
and naturally dropout or withdrawal from the research. Secondly, some
subjects do not experience the event by the end of the study, but
they would have experienced the event if they keep being
observed. Finally, the researchers lose touch with the subjects in the
middle of the investigation. In practice, since these subjects carry
partial information about survival, the scientists will label the
above circumstances as censored observations
\cite{bewick2004statistics} instead of ignoring them and decreasing
the sample size.


In clinical trials, Kaplan-Meier estimator
\cite{kaplan1958nonparametric}, a non-parametric analysis, is a
commonly applied statistical method in the survival analysis for the
measurement of the fraction of the survival time after the treatment
\cite{aalen2008survival} and for generating the corresponding survival
curve \cite{altman1990practical}. Kaplan-Meier estimator also works
well with the mentioned three difficult situations under the
consideration of various assumptions \cite{etikan2017kaplan}
\cite{goel2010understanding}. First of all, the event status only
consists of two mutually exclusive states: censored (0) or failure
(1).  Secondly, the censored observations have identical survival
prospects as those who keep being observed. Thirdly, the objects have
equal survival probabilities no matter they participate in the
research early or late. Lastly, the survival time, the time to an
event or censorship, should be clearly defined and precisely recorded
in the study.



Let $0 < t_1 < t_2 < ...$ be the monotonously increasing observed
times, or the number of steps taken by the particle hitting the
absorbing boundary, in the numerical simulations. Let $n_i$ be the
number of particles, who have not yet stop the simulation up to $t_i$
or be absorbed exactly at $t_i$. Let $d_i$ the number of particles
hitting the target boundary at time $t_i$. The Kaplan-Meier or
product-limit estimator $\widehat{S}(t)$ of the survival function
$S(t)$ of the numerical simulation is \cite{aalen2008survival}

\begin{equation}
  \widehat{S}(t) = \prod_{i:t_i \leq t} (1 - \frac{d_i}{n_i})
\end{equation}



\subsubsection{Confidence Interval}


The upper and lower $(1-\alpha) \times 100 \%$ confidence intervals of
the survival function $S(t)$ for a fixed time $t$ was firstly proposed
by Greenwood in 1926 \cite{greenwoodnatural},


\begin{align} \label{eq:greenwood}
  \widehat{S}(t) \pm z_{\alpha / 2} \sqrt{\widehat{Var}[\widehat{S}(t)]}
  \qquad\text{where}\\
  \widehat{Var}[\widehat{S}(t)] = \widehat{S}(t)^2\sum_{t_i \leq t} \frac{d_i}{n_i(n_i - d_i)}
\end{align}

Note, $z_{\alpha}$ is the $\alpha -$th quantile of the normal distribution.

In 1999, Hosmer and Lemeshow \cite{hosmer2011applied} developed the
exponential Greenwood formula based on the earlier works
\cite{kalbfleisch2011statistical}, which provides an asymmetric
confidence interval for $S(t)$

\begin{align} \label{eq:exp_greenwood}
  e^{-e^{c_{+}(t)}} < S(t) < e^{-e^{c_{-}(t)}}
  \qquad\text{where}\\
  c_{\pm}(t) = log(-log\widehat{S}(t)) \pm z_{\alpha / 2} \sqrt{\widehat{V}}
  \qquad\text{and}\\
  \widehat{V} = \frac{1}{(log\widehat{S}(t))^2} \sum_{t_i \leq t} \frac{d_i}{n_i(n_i - d_i)}
\end{align}

Note, if $c1<c2$, there has $e^{-e^{c_2}} < S(t) < e^{-e^{c_1}}$.


Compared with the traditional calculations of the Greenwood confidence
interval, the exponential Greenwood formula will make sure that the
endpoints in Eq.~\ref{eq:exp_greenwood} lie in $(0, 1)$, while the
endpoints in Eq.~\ref{eq:greenwood} could be negative or larger than
$1$ \cite{sawyer2003greenwood}.





\subsubsection{Scaling Relationship}


Particles' average one-step displacement $\triangle l$ in the
fixed-time step Monte Carlo simulations, LRWs, is associated with the
time step is $\delta$:

\begin{equation}\label{eq:scaling_relationship}
  \triangle l = 2 \sqrt{D \delta}
\end{equation}
where $D$ is the diffusion coefficent.

Eq.~\ref{eq:scaling_relationship} implies that the time step $\delta$
must be designed small enough to make sure that $\triangle l$ is
shorter than the smallest geometrical features of the
boundaries. Thus, $\triangle l$ should equal or be less than one-pixel
size in the simulations. Furthermore, the $\delta$ is regarded as a
fundamental bridge between particles' number of steps $t$ and unitless
continuous-time $\tau$,

\begin{equation}\label{eq:tau}
 \tau = t \delta = \frac{(\triangle l)^2 t}{4D}
\end{equation}
where $D$ is 1.

For example, when running the LRWs in the annulus, $\triangle l$ is always
$\frac{1}{100}$ since particle's step length is as same as one-pixel
size. 



\subsubsection{Two-Sample Statistical Tests}

The differences between the survival curves generated by the
Kaplan-Meier estimator are visible sometimes. However, the
dissimilarities won't be easily detected by eyes if the survival
curves are overlapping over some parts or crossing at some time
points.  Since the Kaplan-Meier estimator does not provide any
information on whether two groups of survival data are statistically
similar or different, some popular statistical tests used specially in
the survival analysis course are presented in this section. Which test
should be selected in a specific circumstance is always debated
because there is a fine line between the statistical tests in the
survival analysis. Therefore, acknowledging the data in hand and
identifying the assumptions well is a prerequisite to determine the
tests appropriately.

Before listing the pros and cons of several statistical tests, the
censored survival times will be recalled firstly, which indicates the
time at which a subject is unobserved and the time to the event of a
subject is not recorded \cite{etikan2018choosing}. In this thesis, it
is possible to appear the censoring observation in the beginning or at
any other moment during the Monte Carlo simulations. If the simulation
finished, but the particle did not reach the target boundary, the
particle will be regarded as a right-censored. When the particle is
abandoned and not been observed during the simulations, it is termed
the random right censoring \cite{etikan2018choosing}. Another cause of
a deficient observation of particles' survival times is the left
censoring, which hints that the particles had stopped diffusing before
the simulation began. For instance, the particle is generated in or on
the pixels of roots. As mentioned in the last section, the
Kaplan-Meier method can still cope well with the right-censored and
left-censored observations in output.


In survival analysis, as the time interval gets close to $0$, the
instantaneous hazard rate can be calculated by limiting the number of
events per unit time divided by the number of events at risk
\cite{case2002interpreting}. The hazard ratio is an estimate of the
hazard rate in one group relative to that in another group
\cite{singh2011survival}. If the survival curves are parallel with the
identical shape, the hazard ratio is constant at any interval of
time. In this situation, the log-rank tests, also named the
Mantel-Haenszel, are reliable \cite{custodio2007diagnostics}.

If the hazard ratio does not satisfy the assumption, the log-rank test
will not be powerful to detect the differences in the survival
functions. In such a case, the Gehan-Breslow-Wilcoxon test, also
called Gehan’s generalized Wilcohon procedure, should be considered
alternatively \cite{agarwal2012statistics}. Also, under the constant
hazard ratio assumption, the Wilcoxon tests might be more reliable
than the log-rank tests \cite{custodio2007diagnostics}. The former one
gives more weight to the early failures, but the latter one is more
suitable for comparing the later events in the data
\cite{custodio2007diagnostics}. Generally, some general non-parametric
tests, based on the rank ordering (e.g. Mann-Whitney U test,
Kruskal-Wallis, etc.), are not always feasible in censoring survival
data \cite{agarwal2012statistics}. However, Gehan’s generalized
Wilcohon test is still robust when the censoring rates are low, and
the censoring distributions of groups are equal
\cite{karadeniz2017examining}.


Neither log-rank test nor Gehan’s generalized Wilcohon test can work
well when the survival curves cross while the Tarone-Ware test should
be chosen \cite{leton2001equivalence}. It pays more attention to the
failures happening somewhere in the middle of study
\cite{etikan2017kaplan}. Moreover, there is no limitation of the
number of groups when the Tarone-Ware test is applied
\cite{custodio2007diagnostics}. Similarly, the Fleming-Harrington test
is also accessible and robust for testing the differences between two
or more survival curves in the right-censored data based on the
counting process \cite{harrington1982class}.



\end{comment}
