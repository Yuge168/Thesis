\subsection{\highlight[id=Yuge, comment={Dave, please have a look at this section and give me feedback. Thanks!}]{Limitations in Practice}}



In this thesis, the heat equation defined in $2$-dimensional domain
$\Omega$, which is bounded by the border of the image and the whole
root system. $\Omega$ has millions of pixels, the extremely complex
roots and various boundary conditions. For calculating the heat
content contained in $\Omega$, the numerical computational techniques
can be used to approximate the solutions of the heat equation, but
some practical difficulities have to be considered since all of them
have an intrinsically similar feature - mesh discretization in the
time and space dimension. For instance, the far more efforts are
required in sovling heat equation by FDM and FVM because of the
complicated boundary of the roots and non-continuous issues. Although
the whole $2$-dimensional root image can be regarded as a discretized
domain, it is still time-consuming and challenging to trace and
identify the boundary of roots, label the nodes, and generate the
coordinates and connectivities among the nodes in the preprocessing
stage of FEM. The finer discretization, the more accurate
approximation of the original IBVP and the higher cost of computation
time in the numerical methods. Moreover, the final goal of this
research is not the numerical approximation of the temperature
distribution over time and space, but the heat content
$Q_{\Omega}(t)$, defined as an integration over the space dimension,
which results in the extra effort and errors.



Numerical methods provides globle solution to the PDEs by determining
the solution at everywhere or a large number of points even if the
solution is needed at a single point. the computer codes for solution
are relatively complex and may require extensive preprocssing to
define a particular problem in a specified format. the numeical
algorithm become unstable in some case. the errors caused by the
discretizedation of the domain of the intefgration and the numerical
integration methods in the analysis cannot be bounded. the field
solution must be calcuilayedwben id the solution is needed at a single
point or a small collection of points in the domainhomogeneous and
isotropic
