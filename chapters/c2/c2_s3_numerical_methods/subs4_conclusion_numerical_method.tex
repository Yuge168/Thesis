\subsection{Limitations in Practice}



In this thesis, the heat equation defined in $2$-dimensional domain,
which is bounded by the border of the image and the whole root system,
with millions of pixels, the extremely complex roots and various
boundary conditions. Before the calculation of the heat content
contained in the domain, the numerical computational techniques can be
used to approximate the solutions of the heat equation, but some
practical difficulities have to be considered since all the described
numerical methods have an intrinsically similar feature - mesh
discretization in the time and space dimension. For instance, the far
more efforts are required in sovling heat equation by FDM and FVM
because of the complicated boundary of the roots and non-continuous
issues. Although the whole $2$-dimensional root image can be regarded
as a discretized domain, it is still time-consuming and challenging to
trace and identify the boundary of roots, label the nodes, and
generate the coordinates and connectivities among the nodes in the
preprocessing stage of FEM. The finer discrtization, the more accurate
approximation of the original IBVP in the numerical methods. More
importantly, the heat content defined as the integration of the
numerical solution of the heat equation over the space dimension
should also be approximated numerically, which results in the extra
effort and errors.
