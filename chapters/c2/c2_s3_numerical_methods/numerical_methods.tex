\section{\highlight[id=Yuge, comment={Dave, please have a look at this section and give me feedback. Thanks!}]{Numerical Methods for Solving Parabolic Partial Differential Equations}}\label{numerical_methdos}


The heat equation is a crucial time-dependent parabolic partial
differential equations (PDE) characterizing how a quantity diffuses
through a given domain over time. From the physical interpretation,
its solution describes the heat distribution or temperature varying in
time and positions and can be obtained uniquely by considering
specific initial and boundary conditions. When the diffusion
coefficient is constant, the general solution of the heat equation is
in one of the two standard forms \cite{crank1979mathematics}. One is
constituted of a series of error functions or related integrals, which
helps evaluate the diffusion behaviour numerically in the early
stage. Another one is in the form of a trigonometric series, which
converges rapidly for a long time. If the heat equation is defined in
a cylinder, a series of Bessel functions will replace the
trigonometric series. 


As introduced in the section~\ref{traditinal_heat_content}, the heat
content $Q_{\Omega}(t)$, defined as the total amount of heat in a
bounded domain $\Omega$, is the integration of the solution to the
heat equation over the space-dimension. The geometric features of
$\Omega$ can be obtained from the coefficients of the asymptotic
expansion of $Q_{\Omega}(t)$. However, so far, only the first few
terms are available explicitly. Furthermore, the traditional
analytical techniques for solving the heat equation has many
restrictions, and its applications to practical problems will exhibit
difficulties. Firstly, the numerical evaluation of the analytical
solutions is usually by no means trivial because they are in the form
of infinite series. Secondly, either irregular geometries or
discontinuities lead to the complexities, so the explicit algebraic
solutions are close to non-existed. Thirdly, the pure analytical
techniques can apply strictly only to the linear form of the boundary
conditions and to constant diffusion properties
\cite{crank1979mathematics}.


Therefore, numerical techniques and simulations are more practical and
applicable to find solutions to the partial differential equations
(PDEs) than the pure analytical methods. The methods for solving
initial-boundary value problems (IBVPs) based on numerical
approximations have existed for a long time and been developed
considerably, including the finite-difference method (FDM), finite
element method (FEM), finite volume method (FVM), boundary element
method (BEM), and so forth.





