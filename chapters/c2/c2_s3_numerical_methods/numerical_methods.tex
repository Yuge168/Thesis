\section{Numerical Methods for Solving Parabolic Partial Differential Equations}\label{numerical_methdos}


The heat equation is a critical time-dependent parabolic partial
differential equation characterizing how a quantity diffuses through a
given region over time. From the physical interpretation of the heat
equation, its solution describes the heat distribution or temperature
varying in time and positions and can be obtained uniquely by
considering specific initial and boundary conditions. As described in
the section~\ref{traditinal_heat_content}, this thesis aims to
calculate the asymptotic expansion of the heat content, defined as the
integration of the solution over the space-dimension, for the shape
description of a bounded domain.  The general solution of the heat
equation is in one of the two standard forms
\cite{crank1979mathematics}. One is constituted of a series of error
functions or related integrals, which is most suitable for evaluating
short-time diffusion behaviour numerically. Another is in the form of
a trigonometrical series, which converges rapidly for a long time. If
the heat equation is defined in a cylinder, a series of Bessel
functions will replace the trigonometrical series.

However, the traditional analytical techniques for solving the heat
equation has many restrictions, and its applications to practical
problems will exhibit difficulties. Firstly, the numerical evaluation
of the analytical solutions is usually by no means trivial because
they are in the form of infinite series. Secondly, either irregular
geometries or discontinuities lead to the complexities, so the
explicit algebraic solutions are close to non-existed. Thirdly, the
purely analytical techniques can apply strictly only to the linear
form of the boundary conditions and to constant diffusion properties
\cite{crank1979mathematics}.

Therefore, numerical methods and computer simulations are more helpful
and applicable to find solutions to the partial differential equations
(PDEs) than calculating pure analytical solutions. The techniques for
solving initial-boundary value problems (IBVPs) based on numerical
approximations have existed for a long time and been developed
considerably including the finite-difference method (FDM), finite
element method (FEM), finite volume method (FVM), boundary element
method (BEM), and so forth.





