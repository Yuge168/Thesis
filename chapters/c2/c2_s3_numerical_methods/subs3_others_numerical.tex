\subsection{Other Numerical Techniques}


Another method closely related to the FEM is the finite volume method
(FVM). It converts the original heat equation into the integral
forms \cite{eymard2000finite}. However, the accuracy of FVM is related
to the integration with respect to the time and space. Unlike the
domain-type methods (e.g. FDM, FEM, FVM, etc.), the boundary element
method (BEM) transforms the heat equation, defined in a given domain,
into an integral equation over the boundary of the domain using the
boundary integral equation
method \cite{attaway1991boundary}. Especially, when the domain extends
to infinity or the boundary is complex, BEM is more efficient in
computation than other methods because of the smaller surface or
volume ratio \cite{katsikadelis2002boundary} since it only discretizes
the boundary and fits the boundary values into the integral
equation \cite{ang2007beginner}. However, the matrics resulted in BEM
are generally unsymmetric and fully populated, which are difficult to
be solved \cite{mushtaq2010advantages}.


