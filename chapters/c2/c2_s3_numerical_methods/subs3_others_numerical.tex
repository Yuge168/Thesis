\subsection{\highlight[id=Yuge, comment={Dave, please have a look at this section and give me feedback. Thanks!}]{Other Numerical Techniques}}


Another method closely related to the FEM is the finite volume method (FVM) since its fundamental idea is to divide the computational region into a system of independent control volumes. Around each grid point of the whole domain, there has a control volume. A set of discrete equations will be obtained by integrating the PDE over each volume \cite{eymard2000finite}. However, the accuracy of FVM is related to the integration over time and space. Unlike the domain-type methods
(e.g. FDM, FEM, FVM, etc.), the boundary element method (BEM) transforms the heat equation into an integral equation over the
boundary of the domain using the boundary integral equation method
\cite{attaway1991boundary}. When the region extends to infinity or its
boundary is complex, BEM is more efficient in computation than other
methods because of the smaller surface or volume ratio
\cite{katsikadelis2002boundary} since it only discretizes the boundary
and fits the boundary values into the integral equation
\cite{ang2007beginner}. The reduction of the dimensionality of the problem is the basic advantage of BEM. However, the matrics generated in the BEM are
generally unsymmetric and fully populated, which are difficult to be
solved \cite{mushtaq2010advantages}.




