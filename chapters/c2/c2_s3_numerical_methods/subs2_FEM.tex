\subsection{Finite Element Method}


Unlike the FDM, the finite element method
(FEM) \cite{zlamal1968finite} divides the complicated geometries,
irregular shapes, and boundaries into the union of smaller and simpler
subdomains (eg. lattice, triangle, curvilinear polygons, etc.), which
are called finite elements \cite{logan2011first}. The smaller size of
the finite element mesh, the more accurate the approximate
solution. Therefore, FEM has great flexibilites or
adaptivities \cite{reddy1993introduction}. For example, FEM can
provide higher fidelity or higher accuracy in a local region and keep
elsewhere the same. Each subdomain is locally represented by the
element equation, continuous piecewise shape functions, which are
finally assembled into a larger system of algebraic equations for
modelling the entire problem. FEM aims to approximate the numerical
solution by minimizing the associated error function to meet certain
specification of the accuracy, which can be done by the
parallelization. Nevertheless, FEM heavily relies on the numerical
integration, where the quadrature rules sometimes cause
difficulties. FEM requires an amount of human involvement in the
process of building the FE model, checking the result, detecting and
updating the model design. Moreover, compared with FDM, FEM demands a
longer execution time and a larger amount of input data.

