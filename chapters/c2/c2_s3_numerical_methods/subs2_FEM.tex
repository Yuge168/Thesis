\subsection{\highlight[id=Yuge, comment={Dave, please have a look at this section and give me feedback. Thanks!}]{Finite Element Method}}


Unlike the FDM, the finite element method (FEM)
\cite{zlamal1968finite} divides the complicated geometries, irregular
shapes, and boundaries into the union of smaller and simpler
subdomains (e.g. lattice, triangle, curvilinear polygons, etc.), which
are called finite elements \cite{logan2011first}. The smaller size of
the finite element mesh, the more accurate approximate for the
solution. Moreover, FEM has great flexibilities or adaptivity
\cite{reddy1993introduction}. For example, FEM can provide higher
fidelity or higher accuracy in a local region and keep other
subdomains the same. Each subdomain is locally represented by the
element equation, the continuous piecewise shape functions, which are
finally assembled into a system of algebraic equations for modelling
the entire problem. For a considerable number of elements, the
parallelling computation can approximate the solution numerically by
minimizing the associated error function. Nevertheless, FEM heavily
relies on numerical integration, where the quadrature rules sometimes
cause difficulties. FEM requires an amount of human involvement in
building the FE model, checking the result, detecting and updating the
model design. Moreover, FEM demands a longer execution time and an
enormous amount of input data compared with FDM.

