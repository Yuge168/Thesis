
\subsection{Finite Difference Method}

FDM is frequently utilized to converting the heat equation into a
system of algebraically solvable equations
\cite{grossmann2007numerical}. The basic idea is to replace the
derivatives in the equation by the difference quotients. For example,
the FTCS (Forward Time Centered Space) scheme
\cite{pletcher2012computational} discretizes the Laplace operator in
space and the time derivative, and then implements the boundary
conditions on the staggered grid for representing the original
continuous problem.


Let $u(x, y, t)$ be the heat distribution at position $(x, y)$ and
time $t$ in a $2-$dimensional homogeneous and isotropic domain $\Omega$. It is
well-known that without any internal heat sources in the domain, $u(x,
y)$ satisfies the heat equation

\begin{equation}\label{eq:Cartesian_heat_equation}
  u_t = D (u_{xx} + u_{yy})
\end{equation}

Note, $D$ is a constant diffusion coefficient and $u_t$ inidicates
partial derivative with respect to time $t$, while $u_{xx}$ and
$u_{yy}$ indicate second partial derivative with respect to $x$ and
$y$ repectively.

Before the implementation of FTCS, let descrtize $\Omega$ along the
$x-$axis and $y-$axis as a regular lattice. In other words, both the
range of $x$ and that of $y$ are divided into equal intervals
$\triangle l$. Also, the time is devided into equal interval
$\delta$. Let the corrdinates of a representative grid point $(x, y,
t)$ be $(i \triangle l, j \triangle l, n \delta)$, where $\triangle l$
is the distance between two neighboring sites of the lattice and
$\delta$ is the time step. For simplicity, we denote the value of $u$
at the point $(i \triangle l, j \triangle l)$ at time $n \delta$ by
$u(i, j, n)$.

The difference formula for time derivative is

\begin{equation}\label{eq:time_difference}
  u_t = \frac{u(i, j, n+1) - u(i, j, n)}{\delta} + \mathcal{O}(\delta)
\end{equation}

The difference formula for the spatial derivaive of $x$ and $y$ are

\begin{align}
  u_{xx} = \frac{u(i-1, j, n) - 2 u(i, j, n) + u(i+1, j, n)}{(\triangle l)^2} + \mathcal{O}((\triangle l)^2) \label{eq:x_difference} \\
  u_{yy} = \frac{u(i, j-1, n) - 2 u(i, j, n) + u(i, j+1, n)}{(\triangle l)^2} + \mathcal{O}((\triangle l)^2) \label{eq:y_difference}
\end{align}

Dropping the error terms $\mathcal{O}(\delta)$ and
$\mathcal{O}((\triangle l)^2)$ and substituting the
Eq.~\ref{eq:time_difference}, Eq.~\ref{eq:x_difference}, and
Eq.~\ref{eq:y_difference} into original heat equation
Eq.~\ref{eq:Cartesian_heat_equation}, there will have

\begin{equation}\label{eq:FDM_heat_equation}
  \frac{u(i, j, n+1) - u(i, j, n)}{\delta} = D (\frac{u(i-1, j, n) - 2 u(i, j, n) + u(i+1, j, n)}{(\triangle l)^2} + \frac{u(i, j-1, n) - 2 u(i, j, n) + u(i, j+1, n)}{(\triangle l)^2})
\end{equation}

Rearranged Eq.~\ref{eq:FDM_heat_equation} as

\begin{equation}\label{eq:rearrange_FDM}
  u(i, j, n+1) = \frac{D\delta}{(\triangle l)^2} (u(i-1, j, n) - 2 u(i, j, n) + u(i+1, j, n) + u(i, j-1, n) - 2 u(i, j, n) + u(i, j+1, n)) + u(i, j, n)
\end{equation}

Finally, the value of $u(i, j, n+1)$ can be expressed explicitly in
terms of $u(i-1, j, n)$, $u(i+1, j, n)$, $u(i, j-1, n)$, $u(i, j+1, n)$, and $u(i, j, n)$ by

\begin{align}\label{eq:final_FDM_heat_equation}
  u(i, j, n+1) &= \beta (u(i-1, j, n) + u(i+1, j, n) + u(i, j-1, n) + u(i, j+1, n)) \\
               & + (1-4\beta) u(i, j, n) \\
  \beta &= \frac{D\delta}{(\triangle l)^2}
\end{align}


The FTCS is conditionally stable \cite{pletcher2012computational}
because the explicit formula in Eq.~\ref{eq:final_FDM_heat_equation}
is stable if and only if $\beta \leq \frac{1}{2}$, which means

\begin{equation}\label{eq:stable_condition}
  \delta \leq \frac{(\triangle l)^2}{2D}
\end{equation}

Eq.~\ref{eq:stable_condition} implies that if the spatial resolution
$\triangle l$ becomes doubled, the time-step $\delta$ should be
reduced by a factor of four to maintain the numerical stability, which
causes the extremely tiny time-step in the high-resolution
calculations. Moreover, there are three kinds of errors needed to be
considered when using FDM. First of all, in the derivation of the
finite-difference equations, the higher-order terms in the Taylor
series are neglected, constituting the truncation error. If the time
and space interval tends to $0$, the truncation errors will approach
$0$, or the FDM is incompatible or inconsistent with the original heat
equation \cite{crank1979mathematics}. Another class of error appearing
in FDM, called round-off error, results from the loss of precision due
to the computer rounding of decimal
quantities. \cite{hoffman2018numerical}. The last type of error is the
discretization error, which can be reduced by decreasing the time
size, grid size, or both \cite{crank1979mathematics}. Moreover, DFM
becomes less accurate and hard to implement when the problem is
defined in the irregular geometries since the heat equation must be
transformed before applying the Taylor series.


