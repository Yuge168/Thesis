
\subsection{\highlight[id=Yuge, comment={Dave, please have a look at this section and give me feedback. Thanks!}]{Finite Difference Method}}

FDM is a frequently utilized technique to convert the heat equation
into a system of algebraically solvable equations
\cite{grossmann2007numerical}. The basic idea is to replace the
derivatives in the equation by the difference quotients. For example,
the FTCS (Forward Time Centered Space) scheme
\cite{pletcher2012computational} discretizes the Laplace operator in
space and the time derivative. And then, it implements the boundary
conditions on the staggered grid for representing the original
continuous problem.


Let $u(x, y, t)$ be the temperature distribution at position $(x, y)$
and time $t$ in a $2-$dimensional homogeneous and isotropic domain
$\Omega$. It is well-known that without any internal heat sources in
$\Omega$, $u(x,y)$ satisfies the heat equation

\begin{equation}\label{eq:Cartesian_heat_equation}
  u_t = D \left(u_{xx} + u_{yy}\right)
\end{equation}

Note, $D$ is a constant diffusion coefficient and $u_t$ inidicates
partial derivative with respect to time $t$, while $u_{xx}$ and
$u_{yy}$ indicate second partial derivative with respect to $x$ and
$y$ repectively.

Before applying of FTCS, let descrtize $\Omega$ along the $x-$axis and
$y-$axis as a regular lattice. Both the range of $x$ and $y$ are
divided into equal intervals $\triangle l$. Also, the time is devided
into equal interval $\delta$. Let the corrdinates of a representative
grid point $(x, y, t)$ be $(i \triangle l, j \triangle l, n \delta)$,
where $\triangle l$ is the distance between two neighboring sites of
the lattice and $\delta$ is the time step. For simplicity, we denote
the value of $u$ at the point $(i \triangle l, j \triangle l)$ at time
$n \delta$ by $u(i, j, n)$.

By applying the Taylor's series \cite{taylor1715methodus}, the
difference formula for time derivative is

\begin{equation}\label{eq:time_difference}
  u_t = \frac{u(i, j, n+1) - u(i, j, n)}{\delta} + \mathcal{O}(\delta)
\end{equation}

Similarly, the difference formula for the spatial derivaive of $x$ and
$y$ are

\begin{align}
  u_{xx} = \frac{u(i-1, j, n) - 2 u(i, j, n) + u(i+1, j, n)}{(\triangle l)^2} + \mathcal{O}\left((\triangle l)^2\right) \label{eq:x_difference} \\
  u_{yy} = \frac{u(i, j-1, n) - 2 u(i, j, n) + u(i, j+1, n)}{(\triangle l)^2} + \mathcal{O}\left((\triangle l)^2\right) \label{eq:y_difference}
\end{align}

Dropping the error terms $\mathcal{O}(\delta)$ and
$\mathcal{O}((\triangle l)^2)$ and substituting the
Eq.~\ref{eq:time_difference}, Eq.~\ref{eq:x_difference}, and
Eq.~\ref{eq:y_difference} into original heat equation
Eq.~\ref{eq:Cartesian_heat_equation}, there will have

\begin{equation}\label{eq:FDM_heat_equation}
  \frac{u(i, j, n+1) - u(i, j, n)}{\delta} = D\left(\frac{u(i-1, j, n) + u(i+1, j, n) - 4u(i, j, n) + u(i, j-1, n) + u(i, j+1, n)}{(\triangle l)^2}\right)
\end{equation}

Rearranged Eq.~\ref{eq:FDM_heat_equation} as

\begin{equation}\label{eq:rearrange_FDM}
  u(i, j, n+1) = \frac{D\delta}{(\triangle l)^2} \Big( u(i-1, j, n) + u(i+1, j, n) - 4 u(i, j, n) + u(i, j-1, n) + u(i, j+1, n) \Big) + u(i, j, n)
\end{equation}

Finally, the value of $u(i, j, n+1)$ can be expressed explicitly in
terms of $u(i-1, j, n)$, $u(i+1, j, n)$, $u(i, j-1, n)$, $u(i, j+1, n)$, and $u(i, j, n)$ as

\begin{equation}\label{eq:final_FDM_heat_equation}
  u(i, j, n+1) = \beta \Big(u(i-1, j, n) + u(i+1, j, n) + u(i, j-1, n) + u(i, j+1, n) \Big) + (1-4\beta) u(i, j, n)
\end{equation}

Where
\begin{equation}\label{eq:stable_condition}
  \beta = \frac{D\delta}{(\triangle l)^2}
\end{equation}

The FTCS scheme is conditionally stable \cite{pletcher2012computational}
because the explicit formula in Eq.~\ref{eq:final_FDM_heat_equation}
is stable if and only if $\beta \leq \frac{1}{4}$, which means

\begin{equation}\label{eq:stable_condition}
  \delta \leq \frac{(\triangle l)^2}{4D}
\end{equation}

Eq.~\ref{eq:stable_condition} reveals the relationship between the
spatial resolution $\triangle l$ and the time-step $\delta$, which
implies that the high spatial resolution calculations demand an
extremely tiny time-step to maintain numerical stability. Besides,
there are three kinds of errors that should be considered when using
FDM. First of all, the truncation error appears in the process of
ignoring the higher-order terms in the Taylor series to obtain the
finite-difference equations. If the time and space interval tends to
$0$, the truncation errors will approach $0$, or the FDM is
incompatible or inconsistent with the original heat equation
\cite{crank1979mathematics}. Another kind of error in FDM is round-off
error, which results from the loss of precision due to the computer
rounding of decimal quantities. \cite{hoffman2018numerical}. The last
type of error is the discretization error, which can be reduced by
decreasing the time size, grid size, or both of them
\cite{crank1979mathematics}. Finally, DFM becomes inaccurate and
arduous in the practical application when the problem is defined in
the irregular geometries since the heat equation must be transformed
before applying the Taylor series.


