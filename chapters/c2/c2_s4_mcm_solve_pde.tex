\section{\highlight[id=Yuge]{Monte Carlo Methods for Solving Partial Differential Equations}}



Solving nonprobabilistic-type problems by probabilistic-type methods



\subsection{Brownian Motion and Random Walk}


\subsection{Early History of MCMs for PDEs}

In 1928, Courant, Friedrichs, and Lewy published a vital paper, which
introduced the probabilistic interpretation and Monte Carlo algorithms
for the linear elliptic and parabolic partial differential equations
\cite{courant1928partiellen}. In their papers, the original PDEs are
reduced to algebraic equations by replacing the differentials by
difference quotients on the rectilinear mesh of the descritized
region. Based on the rigorous proof, the solution of the difference
equation converges to the solution of the corresponding differntial
equaiotn as the mesh width tends to zero.

\subsection{Probabilistic Representation of PDEs}

\subsubsection{Elliptic PDEs}

\paragraph{Wiener Integral Representaion of the Solution}

For the Dirichlet problem: The solution can be expressed as the
expectation of a functional of a stochastic process, for example,
Brownian motion. The integral representation meets the boundary
condition and has the mean value property. Bounary value problem (BVP) is
colsely related to the first passage time of the path of Brownian
motion. 


For the general elliptic PDE: Wiener integrals to different BVPs can
be generalized via the relationship between elliptic operators,
stochastic differential equations, and the Feynman-Kac formula. The
Wiener integral representation is the expectation with respect to the
path, which are the solution of the stochastic differential equations (SDEs). 


MCMs for the elliptic BVPs:

Simulate the sample paths based on SDEs;
Evaluate the Wiener integrals on the sample path;
Sample until variance is acceptable;


\subsubsection{Parabolic PDEs}


The exact solution of the initial value problem (IVP) to the parabolic
PDEs can be interpret as the average value of a large number of
independent stochastic process.













