\section{\highlight[id=Yuge]{Monte Carlo Methods in Solving Partial Differential Equations}}


Monte Carlo methods (MCMs), the commonly used computational
techniques, aim to generate samples from a given probability
distribution, estimate the functions' expectations under this
distribution, and optimize the complicated objective functions by
using random numbers \cite{kroese2014monte}
\cite{rubinstein2016simulation}. MCMs can be used to solve the IBVPs
by generating the random numbers to simulate the successive positions
of the trajectory of a stochastic process at fixed instants
\cite{kronberg1976solution}\cite{king1951monte}, since the original
continuous problem can be represented by the probabilistic
interpretation and the solution can be approximated by the expectation
of some functional of the trajectories of a stochastic process
\cite{grebenkov2014efficient}\cite{sabelfeld2013random}.


Monte Carlo methods are barely applied in solving parabolic partial
differential equations, such as the $1-$ dimensional and $2-$
dimensional time-dependent heat problem. For example,
Eq.~\ref{eq:final_FDM_heat_equation} is a Smoluchowski equation
\cite{smoluchowski1916drei} and provides a probabilistic
interpretation of the Brownian particle, which is moving randomly one
step to $u(i-1, j)$, $u(i+1, j)$, $u(i, j-1)$, or $u(i, j+1)$, with
probability $\beta$, or remaining at the same position $(i, j)$, with
probability $(1-4\beta)$. As introduced in the papers
\cite{sadiku2006monte} \cite{gemjoz2017mcmheat}, given any $\beta \leq
\frac{1}{4}$, the solution of the $2-$dimensional heat equation at a
specific space-time point can be approximated by averaging the value
of a large number of random-walking particles, whose trajectories or
directions for each step are determined by the random numbers
generated by the Monte Carlo techniques, on the any sites of the
target boundary. However, the drawbacks of this method are
obvious. Firstly, there is the error appeared in the finite-difference
approximation. Secondly, there have statistical sampling errors
inherent in the Monte Carlo simulations. Thirdly, each particle's
detailed trajectory will be simulated until it is absorbed by the
boundary, so it will be time-consuming in the simulation with a high
spatial resolution. Moreover, this method can only estimate the
solution to the heat equation at a particular space-time point.


Without mimicking the detailed trajectories, another efficient random
process in the continuous time and continuous space simulated by MCMs
has been applied frequently in solving the elliptic partial
differential equation, for example, the Laplace’s and Poisson’s
equations \cite{haji1967application} \cite{booth1981exact}
\cite{muller1956some}, and steady-state diffusion equation
\cite{torquato1989efficient}. For example, let $v(\bm{s_0})$ be the
value of the solution to an elliptic PDE at a specific point
$\bm{s_0}$ in the domain $\Omega$ with initial and boundary
conditions. $v(\bm{s_0})$ can be estimated by a point $\bm{s_1}$,
which is sampled uniformly on the largest circle $C_0$ centered at
$\bm{s_0}$ with radius $r_0$ lying entirely in $\Omega$. If $\bm{s_1}$
gets closed to the target bounday within an error, $v(\bm{s_1})$ is
known, and it can be considered as one particle's estimate of
$v(\bm{s_0})$ by multiplying the particle’s statistical weight. If
not, $v(\bm{s_1})$ should be estimated in the same way as
$v(\bm{s_0})$, that is, $\bm{s_2}$ is sampled uniformly on the largest
circle $C_1$ centered at $\bm{s_1}$ with radius $r_1$ lying entirely
in the domain. Check the position of $\bm{s_2}$, and the procedure
will be repeated until the simulation is terminates on the traget
boundary, which is defined as one particle's estimate of
$v(\bm{s_0})$. Finally, averaging a larger number of one-particle
estimates, $v(\bm{s_0})$ will be more accurate. However, this process
has not been used in solving the non-steady-state diffusion problem.
