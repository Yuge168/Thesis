\section{\highlight[id=Yuge, comment={Dave, Please give me the feedback or comments on this subsection.}]{Two-sample Statistical Tests}}\label{statistical tests}

The differences between the survival curves generated by the
Kaplan-Meier estimator are visible sometimes. However, the
dissimilarities won't be easily detected by eyes if the survival
curves are overlapping over some parts or crossing at some time
points.  Since the Kaplan-Meier estimator does not provide any
information on whether two groups of survival data are statistically
similar or different, some popular statistical tests used specially in
the survival analysis course are presented in this section. Which test
should be selected in a specific circumstance is always debated
because there is a fine line between the statistical tests in the
survival analysis. Therefore, acknowledging the data in hand and
identifying the assumptions well is a prerequisite to determine the
tests appropriately.

Before listing the pros and cons of several statistical tests, the
censored survival times will be recalled firstly, which indicates the
time at which a subject is unobserved and the time to the event of a
subject is not recorded \cite{etikan2018choosing}. In this thesis, it
is possible to appear the censoring observation in the beginning or at
any other moment during the Monte Carlo simulations. If the simulation
finished, but the particle did not reach the target boundary, the
particle will be regarded as a right-censored. When the particle is
abandoned and not been observed during the simulations, it is termed
the random right censoring \cite{etikan2018choosing}. Another cause of
a deficient observation of particles' survival times is the left
censoring, which hints that the particles had stopped diffusing before
the simulation began. For instance, the particle is generated in or on
the pixels of roots. As mentioned in the last section, the
Kaplan-Meier method can still cope well with the right-censored and
left-censored observations in output.


In survival analysis, as the time interval gets close to $0$, the
instantaneous hazard rate can be calculated by limiting the number of
events per unit time divided by the number of events at risk
\cite{case2002interpreting}. The hazard ratio is an estimate of the
hazard rate in one group relative to that in another group
\cite{singh2011survival}. If the survival curves are parallel with the
identical shape, the hazard ratio is constant at any interval of
time. In this situation, the log-rank tests, also named the
Mantel-Haenszel, are reliable \cite{custodio2007diagnostics}.

If the hazard ratio does not satisfy the assumption, the log-rank test
will not be powerful to detect the differences in the survival
functions. In such a case, the Gehan-Breslow-Wilcoxon test, also
called Gehan’s generalized Wilcohon procedure, should be considered
alternatively \cite{agarwal2012statistics}. Also, under the constant
hazard ratio assumption, the Wilcoxon tests might be more reliable
than the log-rank tests \cite{custodio2007diagnostics}. The former one
gives more weight to the early failures, but the latter one is more
suitable for comparing the later events in the data
\cite{custodio2007diagnostics}. Generally, some general non-parametric
tests, based on the rank ordering (e.g. Mann-Whitney U test,
Kruskal-Wallis, etc.), are not always feasible in censoring survival
data \cite{agarwal2012statistics}. However, Gehan’s generalized
Wilcohon test is still robust when the censoring rates are low, and
the censoring distributions of groups are equal
\cite{karadeniz2017examining}.


Neither log-rank test nor Gehan’s generalized Wilcohon test can work
well when the survival curves cross while the Tarone-Ware test should
be chosen \cite{leton2001equivalence}. It pays more attention to the
failures happening somewhere in the middle of study
\cite{etikan2017kaplan}. Moreover, there is no limitation of the
number of groups when the Tarone-Ware test is applied
\cite{custodio2007diagnostics}. Similarly, the Fleming-Harrington test
is also accessible and robust for testing the differences between two
or more survival curves in the right-censored data based on the
counting process \cite{harrington1982class}.














