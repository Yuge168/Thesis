\section{\highlight[id=Yuge, comment={Dave, please have a look at this section and give me feedback. Thanks!}]{Monte Carlo Simulation for Approximating Heat Content $Q_{\Omega}(t)$}}

In the section~\ref{numerical_methdos}, several generally utilized
numerical methods \cite{grossmann2007numerical}\cite{zlamal1968finite}
\cite{eymard2000finite} \cite{attaway1991boundary} for approximating
the solutions to the heat equation and their limitations in practice
are presented. In this section, instead of understanding the heat
equation from the macroscopic view, two microscopic probabilistic
interpretations are introduced. Moreover, the Monte Carlo method (MCM)
\cite{rubinstein2016simulation} \cite{kroese2014monte}, one of the
non-deterministic techniques, and its application in solving PDEs
numerically are summarized. Moreover, an alternative fixed-time step
Monte Carlo simulations, lattice random walks (LRWs), is designed. The
most outstanding advantage of the proposed random walk model is that
the heat content can be approximated directly based upon the
probabilistic interpretation of the heat equation.  Finally, the
methods to analyze the output of the Monte Carlo simulations and solve
the sampling-related problems in the simulations are brought up
theoretically.





