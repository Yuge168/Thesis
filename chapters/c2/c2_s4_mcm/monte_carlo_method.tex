\section{Monte Carlo Simulation for Approximating Heat Content}

In the section~\ref{numerical_methdos}, several generally utilized
numerical methods \cite{grossmann2007numerical}\cite{zlamal1968finite}
\cite{eymard2000finite} \cite{attaway1991boundary} for solving the
heat equation, and their limitations in practice are presented. In
this section, one of the non-deterministic algorithms, Monte Carlo
method (MCM) \cite{rubinstein2016simulation} \cite{kroese2014monte},
and its application in approximating the solution of the PDEs are
proposed. As the weaknesses and challenges of applying the numerical
techniques in solving $2-$dimensional heat equation defined in the
real root images with millions of pixels and extremely complex root
systems, the alternative fixed-time step Monte Carlo simulations,
lattice random walks (LRWs), is designed. The most outstanding
advantage of the proposed random walk model is that the integration,
named the heat content, can be approximated directly based upon the
probabilistic interpretation of Brownian motion and the heat equation.
Finally, the methods to analyze the output of the Monte Carlo
simulations and solve the sampling-related problems in the simulations
are brought up theoretically.



