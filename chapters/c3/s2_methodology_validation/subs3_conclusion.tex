
\subsection{Conclusion}


Instead of calculating the asymptotic expansion of the heat content
manually as $\tau \rightarrow 0^+$, the total heat energy $\beta$
\cite{gilkey1994heat} for time $\tau > 0$ can be approximated by the
fixed-time step Monte Carlo simulations for describing the full-scale
geometrical features of the annulus. Moreover, the required
number of particles in the simulations determined by the DKW
inequality is much smaller than the superabundant value estimated by
Chebyshev’s inequality.
