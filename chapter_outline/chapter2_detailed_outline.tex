\documentclass{article}
\usepackage{outlines}
\usepackage{enumitem}
\usepackage{verbatim}
\setenumerate[1]{label=\Roman*.}
\setenumerate[2]{label=\Alph*.}
\setenumerate[3]{label=\roman*.}
\setenumerate[4]{label=\alph*.}
\usepackage[backend=bibtex]{biblatex}
\usepackage{amssymb}
\usepackage{comment}
\usepackage{bm}
\usepackage{amsmath}
\usepackage{amsfonts}
\bibliography{../thesisref.bib}
\usepackage{changes}
\definechangesauthor[name=Yujie Pei, color=green]{Yuge}

\begin{document}

\begin{outline}[enumerate]

  \centering{C2: An Alternative Mathematical Method for Shape Description}

  \1 Kac’s Idea: Can One Hear the Shape of a Drum? \cite{kac1966can}
  
    \2 Interpretations of Kac's Problem
      \3 When the drum vibrates, one can hear the sound, which is composed of tones of various frequencies. How much can shape features be inferred from hearing a discrete spectrum of pure tones produced by a drum?
      \3 If a complete sequence of eigenvalues of the Dirichlet problem for the Laplacian can be obtained precisely, will people determine the shape of a planar?
      
    \2 Problem Statement by Mathematical Language
      \3 Consider a simply connected membrane $\Omega$ in the Euclidean space bounded by a smooth convex curve $\partial \Omega$ (e.g. a drum without any holes)
      \3 Find function $\phi$ on the closure of $\Omega$, which vanishes at the boundary $\partial \Omega$, and a number $\lambda$ satisfying $-\Delta \phi = \lambda \phi$.
        \4 $\Delta$ is the Laplace operator. e.g. $\Delta = \sum_{i=1}^{n} \frac{\partial ^2}{\partial x_i^2}$ in Cartesian coordinate system.
        % In a Cartesian coordinate system, the Laplacian is given by the sum of second partial derivatives of the function with respect to each independent variable.
        
        \4 If there exists a solution $\phi \neq 0$, the corresponding $\lambda$ is defined as a Dirichlet eigenvalue.
        \4 For each domain $\Omega$, there has a sequence of eigenvalues $\lambda_1, \lambda_2, \lambda_3, ... $ corresponding to a set of eigenfunction $\phi_1, \phi_2, \phi_3, ...$.
        \4 $\phi_k$ form an orthonormal basis of $L^2(\Omega)$ of real valued eigenfunctions; the corresponding discrete Dirichlet eigenvalues are positive ($\lambda_k \in \mathbb{R^{+}}$).
      \3 An important function \cite{grieser2013hearing}: $h(t) = \sum_{k=1}^{k=\infty} e^{-\lambda_kt}$
        \4 It is a Dirichlet series.
        \4 It is called the spectral function or the heat trace.
        \4 It is smooth and converges for every $t>0$.

    %\2 Understanding the Problem in the Physical Context by Diffusion Theory
      %\3 Diffusion Equation: It describes the density fluctuations of the diffusing material.
      %\3 $C_{\Omega}(\bm{s}, t | \bm{s_0})$
        %\4 It indicates the concentration of matter (i.e. microscopic particles) at $\bm{s}$ at time $t$ starting from $\bm{s_0}$.
        %\4 It satisfies the diffusion equation \cite{crank1979mathematics}
        %\par
        %\begin{equation}\label{eq:diffusion_eq}
          %\frac{\partial C_{\Omega}(\bm{s}, t| \bm{s_0})}{\partial t} = D\Delta C_{\Omega}(\bm{s}, t| \bm{s_0})
        %\end{equation}
        %\begin{itemize}
          %\item $D$ is the diffusion coefficient
          %\item In Kac's paper, $D$ is a constant. For simplicity, let $D$ be $1$.
        %\end{itemize}
        
        %\4 Initial condition: $C_{\Omega}(\bm{s}, t| \bm{s_0}) = \delta(\bm{s}-\bm{s_0})$ as $t \rightarrow 0$ (Dirac delta function): $\delta \neq 0$ if and only if $\bm{s} = \bm{s_0}$; all of the mass concentrates at the origin;
        %\4 Dirichlet Boundary condition: $C_{\Omega}(\bm{s}, t| \bm{s_0})=0$ for $\bm{s} \in \partial \Omega$
          %\begin{itemize}
            %\item It is also called the absorbing boundary condition.
            %\item e.g. any chemical molecule will be instantly absorbed when it touches the boundary $\partial \Omega$. 
          %\end{itemize}
          
        %\4 It can be expressed in terms of eigenvalues $\lambda_k$ and the corresponding normalized eigenvalues $\phi_k$ by: $C_{\Omega}(\bm{s}, t| \bm{s_0}) = \sum_{k=1}^{\infty} \phi_k(\bm{s}) \phi_k(\bm{s_0}) e^{-\lambda_kt}$.
 
      %\3 As $t \rightarrow 0^{+}$,
      %\par
      %\begin{equation}
        %\begin{split}
        %\int_{\Omega} C_{\Omega}(\bm{s}, t| \bm{s_0})d\bm{s_0} &\simeq \int_{\Omega} C_{\Omega}(\bm{s_0}, t| \bm{s_0})d\bm{s_0}\\
        %&= \int_{\Omega} \Big(\sum_{k=1}^{\infty} \phi^2_k(\bm{s_0}) e^{-\lambda_kt} \Big) d\bm{s_0} \\
        %&= \sum_{k=1}^{k=\infty} e^{-\lambda_kt} \\
        %&= h(t)
        %\end{split}
      %\end{equation}
      %where $\int_{\Omega} \phi^2_k(\bm{s_0})d\bm{\bm{s_0}} = 1$ because of the normalization of $\phi$.
        
    \2 Summarize the Results of Kac's Idea
      \3 As $t \rightarrow 0^{+}$, the leading terms of the asymptotic expansion of $h(t)$ imply the geometrical attributes of $\Omega$
        \4 the total area
        \4 the perimeter
        \4 the curvature
      \3 If the domain $\Omega$ has the polygonal boundary, the third term shows in the information about the interior angles of the polygon \cite{grieser2013hearing}.

   \2 Conclusion
      \3 Advantages
        \4 Kac proposed a novel analytical mathematical method for the shape description without using measuring tools, e.g. rulers.
        \4 Other mathematicians extended Kac's idea in exploring the geometrical information of more complex domains with various boundary conditions \cite{khabou2007shape}\cite{gottlieb1985eigenvalues}\cite{gottlieb1983hearing} \cite{zayed1989heat}\cite{sleeman1984trace}.  
     \3 Limitations
        \4 It is only available for the convex domain, which has a smooth or piecewise smooth boundary.  
        \4 Except in very few cases (i.e. rectangular, disk, certain triangles), the complete sequence of eigenvalues $\lambda_k$ can not be calculated \cite{grieser2013hearing}.
        \4 Only the first few terms in the asymptotic expansion of $h(t)$ are explicitly available.

        \newpage
        
  \1 Extended Work of Kac’s Idea \cite{desjardins1994heat}\cite{vandenberg1994heat}: Heat Content
    \2 Fouier's Heat Equation \cite{baron1878analytical}
      %\3 Connection with the diffusion equation Eq.~\ref{eq:diffusion_eq} mentioned in Kac's paper
        %\4 When the diffusion coefficient of the diffusion equation is independent of the density (i.e. constant diffusion coefficient), the diffusion equation is also named the heat equation.
      \3 Mathematical Formula
        \par
        \begin{equation}\label{eq:heat_eq}
          \frac{\partial u(\bm{s}, t)}{\partial t} = \Delta u(\bm{s}, t)
        \end{equation}
        %\4 $D$ is a positive coefficient. %  called the diffusivity of the medium; Diffusivity is a rate of diffusion, a measure of the rate at which particles or heat or fluids can spread.
      \3 Interpretation
      \par
      It is a deterministic model used to characterize the evolution of quantities over the space and time. (e.g. the flow of heat)
        %\3 Physical interpretation: conservation of heat per unit volume over an infinitesimally samll volumne lying in the interior of the flow domain;

    \2 Summarize the Idea
      \3 Initial-Boundary Value Problem (IBVP)
        \par
        $u(\bm{s}, t)$ indicates the value of the tempretature at $\bm{s} \in \Omega$ at time $t$ satisfying Eq.~\ref{eq:heat_eq} and 
         \4 Initial condition: $u(\bm{s}, t) = f(\bm{s})$ as $t \rightarrow 0$.
         \4 Dirichlet boundary condition: $u(\bm{s}, t)=0$ for $\bm{s} \in \partial \Omega$
         \par
          It is also called the absorbing boundary condition; i.e. any molecule will be instantly absorbed when it touches the boundary $\partial \Omega$;


      \3 A Basic Integration
         \par
          \begin{align}
            \beta_{\Omega}(f, g)(t) &= \int_{\Omega} \int_{\Omega} H_{\Omega}(\bm{s}, t | \bm{s_0}) f(\bm{s_0}) g(\bm{s}) d\bm{s_0} d\bm{s} \label{eq:integral_full} \\
            &= \int_{\Omega} u(\bm{s}, t) g(\bm{s}) d\bm{s} \label{eq:integral_convol}
          \end{align}
         
        \4 $H_{\Omega}(\bm{s}, t | \bm{s_0})$ is called the heat kernel of $\Omega$ describing the density of the heat at $\bm{s}$ after time $t$ when initially there is only one single hot source at $\bm{s_0}$. 
        \4 $u(\bm{s}, t)$ is the general solution to Eq.~\ref{eq:heat_eq}, which can be expressed as the convolution of the initial condition with the heat kernel of the domain.
        \4 $g(\bm{s})$ is an auxiliary test function for studying the distributional nature of the tempretature function $u(\bm{s}, t)$ near $\partial \Omega$. 

     \3 Heat Content Calculation
       \par
       Given 
       \begin{equation}\label{eq:g}
         g(\bm{s}) = 1  
       \end{equation}

       \begin{align}
         Q_{\Omega}(t) &= \int_{\Omega} \int_{\Omega} H_{\Omega}(\bm{s}, t | \bm{s_0}) f(\bm{s_0})  d\bm{s_0} d\bm{s} \label{eq:heat_content_integral_full} \\
            &= \int_{\Omega} u(\bm{s}, t) d\bm{s} \label{eq:heat_content_integral_convol}
       \end{align}
       
          
        
     \3 Shape Characterization
        \4 As $t \rightarrow 0^{+}$, $Q_{\Omega}(t) \simeq \sum_{n=1}^{\infty} \beta_n(\Omega) t ^{\frac{n}{2}}$
        \4 Obtain geometrical information of $\Omega$ from $\beta_n$
         \begin{itemize}
           \item area
           \item length
           \item scalar curvature
           \item mass
         \end{itemize}
         

    \2 Conclusion
      \3 Strengthness
        \4 Instead of calculating a complete sequence of the Dirichlet eigenvalues for exploring the shape attributes of geometry, the asymptotic expansion of the heat content, defined as integrating the solution to the heat equation over the space-dimension, also implies the geometrical characteristics.  
      \3 Limitations
        \4 Only the infinitly differentiable boundary $\partial \Omega$ is considered.
        \4 Only the first few terms in the asymptotic expansion are explicitly known.
        \4 Either irregular geometries or discontinuities lead to the complexities, so the explicit solutions $u(\bm{s}, t)$ are close to non-existed.
        \4 The numerical evaluation of the analytical $u(\bm{s}, t)$ and $Q_{\Omega}(t)$ is usually by no means trivial because they are in the form of infinite series.
        \4 Similiarly, only the first few coeffients $\beta_n$ in the asymptotic expansion of $Q_{\Omega}(t)$ can be expressed as the complicated explicit forms.
        
        \newpage
        

  \1 Numerical Methods for Solving Parabolic Partial Differential Equations
    \2 Parabolic PDEs: to characterize time-dependent phenomena
    \2 The intrinsically similar features of the traditional computational techniques are mesh discretization in time and space.
    \2 Finite Difference Method (FDM) \cite{grossmann2007numerical}
      \3 Basic Ideas: replace derivatives in the equation by the difference quotients; 
      \3 A simpliest example of FDM in solving $2-$dimensional heat equation by Forward Time Centered Space (FTCS) \cite{pletcher2012computational}
         \4 It is an explict method: utilize a simple explicit formula to evaluate the unknown function at each of the spatial mesh points at the new time level.
         \4 Prilimarty step: discretize the domain by a set of mesh points.
         \4 Secondary step: replace the derivatives by finite difference approximations based on FTCS.
           \begin{itemize}
             \item discretize the Laplace operator in space
             \item discretize the time derivative
           \end{itemize}
         \4 Third step: determine the time and step size by a condition for numerical stability.
         \4 Fourth step: fill in the initial and boundary values in the initialied matrix.
         \4 Fifth step: estimate the field values (e.g. tempreature) at the finite number of space-time points by the iterations.
         \4 Final Step: visualize the numerical results.

      \3 Stengthness
         \4 Compared with other numerical methods, it is the easy to code from the implementational point of view.
      \3 Weakness \cite{crank1979mathematics} \cite{hoffman2018numerical}
         \4 The truncation error appears in the process of ignoring the higher-order terms in the Taylor series to obtain the finite difference equations.
         \4 Round-off error results from the loss of precision due to the computer rounding of decimal quantities.
         \4 Discretization error can be reduced by decreasing the time size, grid size, or both of them, but the computational time will be longer.
         \4 It will be inaccurate and arduous in the practical application when the problem is defined in the irregular geometries.
      
    \2 Finite Element Method (FEM) \cite{zlamal1968finite}
      \3 Foundamental Concepts 
         \4 Finite element: divide the complicated geometries, irregular shapes, and boundaries into an union of smaller and simpler subdomains (e.g. lattice, triangle, curvilinear polygons, etc.) \cite{logan2011first}; adjacent element are connected by the nodes.
         \4 Element equation: reprsent each subdomain by the piecewise continuous polynomial basis function.
         \4 Model the whole system: assemble all the element equations into a system of algebraic equations.
         \4 Slove algebraic equations by minimizing the associated error function.
      \3 Stengthness
         \4 FEM can be used to solve a wide range of PDEs defined in a complex geometry.
         \4 Spatial discretization is flexible since the mesh can adapt to irregularly shaped boundaries to reduce geometric errors.
         \4 A specific region can be refined locally to give more resolution.
         \4 For a considerable number of elements, the paralleling computation can be used to improve the computational efficiency.
      \3 Weakness
         \4 Users may make mistakes in building the FE model, checking the result, detecting and updating the model design.
         \4 The round-off errors will affect the precision of the nuemrical results.
         \4 FEM demands a longer execution time and an enormous amount of input data compared with FDM.
      
    \2 Other Tranditional Computational Methods
      \3 Finite Volume Method (FVM) \cite{eymard2000finite}
      \3 Boundary Element Method (BEM) \cite{attaway1991boundary}
      
    \2 Limitation in Practice  
      \3 The size of the mesh, subdomain, or volumn will determine the precision of the approximated solution, the smaller being the better. However, the finer discretization will increase the demand for computational resources such as memory and processor time.
      \3 In this thesis, the problem domain $\Omega$ is bounded by the border of the image and the edge of the whole extremly complicated root system with millions of pixels and various boundary conditions. The complexity of coding is the main practical limitation.
      \3 The purpose of this thesis is to approxiate $Q_{\Omega}(t)$, defined as an integration over the space dimension, which results in the extra effort and errors.
        

    \newpage


  \1 Monte Carlo Simulation for Approximating Heat Content $Q_{\Omega}(t)$

    \2 Background \cite{jacobs2010stochastic}
      \3 Stachasic Differential Equations (SDEs)
        \4 Stochastic process
        \4 Brownian motion
      \3 Connection Between SDEs and Heat Equation 
        \4 Ito calculus
        \4 Intepretating the heat equation by the probability density of particles undergoing Brownian motion


    \2 A Special Case of Heat Content Calculation  
      \3 Uniform initial tempretaure distribution in Eq.~\ref{eq:heat_content_integral_full}
          \par
          \begin{equation}\label{eq:uniform_initial_condition}
             f(\bm{s_0}) = \frac{1}{|\Omega|} 
          \end{equation}
          
           \4 $|\Omega|$ is the area of the domain $\Omega$.
           \4 Particles's initial positions are distributed uniformly in $\Omega$.
          
      \3 Intepretation of Eq.~\ref{eq:g}: observing all the Brownian particles unbiasedly.
           
      \3 Probabilistic Intepretation of Heat Kernel $H_{\Omega}(\bm{s}, t | \bm{s_0})$: conditional probability density function of Brownian particles

      \3 Probabilistic Interpretation of $Q_{\Omega}(t)$: Survival Probability $S_{\Omega}(\tau)$
         \4 First passgae time $\tau$: the time taken by the particle to encounter the absorbing boudary $\partial \Omega$ from the initial position.
         \4 Derivation of $S_{\Omega}(\tau)$ based on $H_{\Omega}(\bm{s}, t | \bm{s_0})$.
        

%   \2 Monte Carlo Integration for Eq.~\ref{eq:heat_content_integral_full}
 %     \3 Introduction     
  %      \4 Definition: utilizing the random sampling of a function to compute an estimate of its integral numerically \cite{hammersley1960monte}.
   %     \4 Uniform Sampling Method
    %  \3 Given an initial position $\bm{s_0} \in \Omega$, approximating $H_{\Omega}(\bm{s}, t | \bm{s_0})$ by simulating Lattice Random Walks (LRWs)
     %   \4   
         
    \2 Monte Carlo Simulation (LRWs) for approximating $S_{\Omega}(\tau)$
      \3 Monte Carlo Integration
        \4 Introduction
          \begin{itemize}
            \item Definition: utilizing the random sampling of a function to compute an estimate of its integral numerically \cite{hammersley1960monte}.
            \item Uniform Sampling Method
          \end{itemize}
       \4 Given an initial position $\bm{s_0} \in \Omega$, approximating $H_{\Omega}(\bm{s}, t | \bm{s_0})$ by simulating the trajectories of a large number of particles by Lattice Random Walks (LRWs).
       \4 Sampling a lager number of particles, whose initial sites are distributed uniformly within $\Omega$ to estimate $S_{\Omega}(\tau)$.
          
     \3 Design LRWs in the $2-$ dimensional image
         \4 Initial condition: uniform distribution within $\Omega$, which is bounded by the border of the image and the edge of the target object.
         \4 Boundary condition
           \begin{itemize}
             %\item Reflecting boundary condition on top and bottom edges of image;
             \item Perodic boundary condition the edges of the image;
             \item Absorbing boundary condition on the boundary of the target shape.
           \end{itemize}
      
        %\4 Efficent Random Walks (ERWs) - Dave's code
             
      \3 Sample Size Determination - Dvoretzky–Kiefer–Wolfowitz (DKW) inequality \cite{dvoretzky1956asymptotic}
        \4 Mathematical formula and interpretation
        \4 Strengths
          \begin{itemize}
            \item Distribution-free
            \item Sample size calculation does not depend on 
              \begin{itemize}
                \item Domain shape and size
                \item Target geometry
                \item B.C.s
              \end{itemize}
          \end{itemize}
          
     \3 Output Analysis (in theory)
        \4 Kaplan-Meier Estimator \cite{kaplan1958nonparametric} \cite{aalen2008survival}
        \4 Confidence Interval \cite{greenwoodnatural} \cite{hosmer2011applied} \cite{kalbfleisch2011statistical}\cite{sawyer2003greenwood}
        \4 Two-Sample Statistical Tests (weighted logrank tests) \cite{custodio2007diagnostics} \cite{agarwal2012statistics} \cite{karadeniz2017examining} \cite{leton2001equivalence} \cite{etikan2017kaplan} \cite{harrington1982class}
          \begin{itemize}
            \item Wilcoxon
            \item Tarone-Ware
            \item Peto
            \item Fleming-Harrington
          \end{itemize}
          



        %\begin{itemize}
            %\item Uniform sampling: random points are distributed uniformly in the domain.
            %\item Stratified sampling: splitting up the original integral into a sum of intgrals over sub-regions.
             % divides the domain into N sub-domains and places a random sample within each of them.
            %\item Importance sampling: choosing samples from a distribution, which has a similar shape as the function being integrated.
          %\end{itemize}
          
    %\3 Multi-dimensional Integration
      % \4 General solution $u(\bm{s}, t) = \int_{\Omega} C_{\Omega}(\bm{s}, t | \bm{s_0}) \delta(\bm{s}-\bm{s_0}) d \bm{s_0}$
       %   \begin{itemize}
        %    \item Background
%              \begin{itemize}
 %                \item Stochastic process:
                %\par
  %              %a collection of random variables indexed by the time variable;
    %            \item Stochastic differential equations (SDEs)
   %                   
     %         \end{itemize}
              
              
      %      \item Local solution $u(\bm{s_0}, t)$

       %       \begin{itemize}
        %        \item Generating a large number of samples starting from $\bm{s_0}$.
         %       \item $u(\bm{s_0}, t)$ can be expressed as an expecation depending on the sample paths of the stochastic process.
          %      \item Independent random sample trajectories of the stocahstic process can be generated by Monte Carlo techniques to approximate the expectation and thus to estimate $u(\bm{s_0}, t)$.
           %   \end{itemize}
%
 %           \item Approximate the global solution $u(\bm{s}, t)$ by estimating the lacal solution at a large number of positions distributed uniformly within $\Omega$.
  %        \end{itemize}
          
   %   \4 Heat content $Q_{\Omega}(t) = \int_{\Omega} u(\bm{s}, t) d\bm{s}$
    %    \begin{itemize}
     %     \item Sampling points uniformly within $\Omega$.
      %    \item Average the samples of $u$ to estimate the numerical $Q_{\Omega}(t)$ 
       %   %\item $Q_{\Omega}(t) \approx |\Omega| \frac{1}{M} \sum_{i=1}^{M} u(\bm{s_i})$
        %\end{itemize}
        
       
    %\2 Probabilistic Interpretation of $Q_{\Omega}(t)$: Survival Probability

    %\3 Background
     %  \4 Brownion motion
      %   \begin{itemize}
       %     \item Description and history: irregualer, continuous, and permanent random motion observed by Brownion: e.g. microscopic pollen grains suspend in the water.
        %     \item Mathematical definition and properties
%                      \begin{itemize}
 %                       \item It is a continuous stochastic process.
  %                      \item For each time $t$, it is a Gaussian random variable with mean zero and variance $t$.
   %                     \item The increments are mutually independent.
    %                    \item The increments are also mean zero Gaussian random variable with variance $t-s$
               %                 \end{itemize}
         %\end{itemize}
         
      %  \4 \highlight[id=Yuge]{Inteprete the deterministic heat equation from the view of Brownian particles??}
       % \4 Connection between the continuous Brownian motion and discrete random walks
       % \4 Connection between the $2-$dimensional LRWs and heat equation from the probalistic prospective \cite{kalinay2011survival}
        %  \begin{itemize}
         %   \item $P_{\Omega}(x, y, \tau + \delta | \bm{s_0})$: conditional probability of a particle to be in position $(x, y) \in \Omega$ at time $\tau + \delta$ starting from $\bm{s_0}$, $(x_0, y_0) \in \Omega$, at time $\tau=0$.
          %  \item %Stochastic difference equation
           %   \par
            %  \begin{multline*}
             %   P_{\Omega}(x, y, \tau + \delta | \bm{s_0}) = \frac{1}{4}\Big( P_{\Omega}(x - \triangle l, y, \tau | \bm{s_0}) + P_{\Omega}(x + \triangle l, y, \tau | \bm{s_0}) \\
              %  + P_{\Omega}(x, y - \triangle l, \tau | \bm{s_0}) + P_{\Omega}(x, y + \triangle l, \tau | \bm{s_0})\Big)
             % \end{multline*}
           % \item When  $\delta \rightarrow 0$, $\triangle l \rightarrow 0$, and $\delta \sim (\triangle l)^2$,
            %  \par
             % \begin{equation}
              %  \frac{\partial P_{\Omega}(x, y, \tau | \bm{s_0})}{\partial \tau} = D \Big( \frac{\partial ^2 P_{\Omega}(x, y, \tau | \bm{s_0})}{\partial x^2} + \frac{\partial ^2 P_{\Omega}(x, y, \tau |\bm{s_0})}{\partial y^2}\Big)
             % \end{equation}
              
          %\end{itemize}
          
      %\3 Survival probability $S_{\Omega}(\tau)$
       %  \4 First passage time $\tau$: the number of steps taken by the particle to encounter the absorbing boudary of $\Omega$.
        % \4 \par
         %     \begin{align}
          %      S_{\Omega}(\tau|\bm{s_0}) &= \int_{\Omega} P_{\Omega}(x, y, \tau | \bm{s_0}) dxdy \\
           %     S_{\Omega}(\tau) &= \int_{\Omega} S_{\Omega}(\tau|\bm{s_0}) d\bm{s_0}
            %  \end{align}
              
       %\4 $S_{\Omega}(\tau)$ and $Q_{\Omega}(t)$
         %  \par
        %   since the initial position $\bm{s_0}$ is distributed uniformly within $\Omega$, $S_{\Omega}(\tau) \simeq Q_{\Omega}(t)$.





        



\end{outline}

\newpage

\printbibliography


\end{document}
