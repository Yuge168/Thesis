\documentclass{article}
\usepackage{outlines}
\usepackage{enumitem}
\usepackage{verbatim}
\setenumerate[1]{label=\Roman*.}
\setenumerate[2]{label=\Alph*.}
\setenumerate[3]{label=\roman*.}
\setenumerate[4]{label=\alph*.}
\usepackage[backend=bibtex]{biblatex}
\usepackage{amssymb}
\usepackage{comment}
\usepackage{bm}
\usepackage{amsmath}
\usepackage{amsfonts}


\begin{document}

\begin{outline}[enumerate]

  \centering{C2 S4: Monte Carlo Methdod for Estimating Heat Content}

  
  \1 Monte Carlo Integration
   %\2 Diffusion process, BM and stachastic differential equation (SDE)
     \2 Definition and application
     \2 Heat content $Q_{\Omega}$
        \3 Golbal solution $u(\bm{s}, t)$: integration over the initial position: conditional probability multiply initial condition
           \4 Estimating the local solution based on the Random Walk Method (RWM)
           \4 Generating initial positions distrbuted uniformly in the domain, and run RWM to estimate  $u(\bm{s}, t)$.
        \3 Intergating $u(\bm{s}, t)$ over the space domain

     
  %\1 Monte Carlo Methods (MCMs)
    %\2 Definition
    %\2 Properties and application

  \1 Brownian Motion (BM)
    \2 General description and history of BM (easier for non-mathematican to understand)
      \3 irregualer, continuous, and permanent random motion found by Brownion: microscopic pollen grains suspend in the water
      \3 Einstein's explaination for BM and the solution to the heat equation
         \4 Becaue of the continual collision from the surrounding water molecules, pollen grains have the same average kinetic energy as the molecules.
         \4 Einstein's proof: BM provides a solution to the Fourier's heat equation    
    \2 Mathematical Prospective
      \3 Formula   
      \3 BM's propertities
    \2 Random Walk
      \3 Definition and history
      \3 Connection with BM

     
    %\2 Local solution:
      %\3 It can be expresssd as the expectation dependening on the sample path of a stocahstic process (diffusion process) defined by the SDE.
      %\3 The expectations can not be obtained analytically, but they can be estimated from averaging samples of diffusion process generated by the MCMs. The samples undergoing the Brownian motion independently and their behaviors satisify the SDE.
    %\2 Strengthness of RWM
      %\3 simple to program
      %\3 the estimated value to the solution is stable and accurate
      %\3 available for parallelling computation
   % \2 Disadvantages
      %\3 Only get local solution  

  %\1 Probabilistic interpretation of heat equation by Lattice Random Walks (LRWs)??

\end{outline}

\end{document}


