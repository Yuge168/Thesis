

%________________________________________________

\subsection{Circle and Rectangle}

   \subsubsection{Image Description}
      \begin{itemize}
        \item Image size: $1200 \times 1000$ pixels
        \item Surface area of shapes: $90000$ pixels
        \item The centroid of the shape is located at the center of the image
      \end{itemize}
      
   \subsubsection{Purposes: Methodology Validation}
       \par 
       Given two distinct convex geometries, whether
       \begin{itemize}
         \item the behaviours of the survival function of LRWs are consistent with the theoretical results \cite{DesjardinsS1994HAFO}.
         \item survival curves can be used to describe and distinguish them.
       \end{itemize}
       

   \subsection{Complicated Branching Structures}
       
     \subsubsection{Image Description}
       \begin{itemize}
         \item Image size: $1200 \times 1000$ pixels
         \item Surface area of shapes: $90000$ pixels
         \item Iterate the template $3, 4, 5, 6$ times to produce the targeted branching geometries labelled as $L_3, L_4, L_5, L_6$.
         \item Three groups of images labelled as $G_1, G_2$

           \begin{itemize}
             \item $G_1$: the target object $G_1 L_i$ ($i=3, 4, 5, 6$) is equidistant to the edges of an image.
             \item $G_2$: the template of $G_2 L_i$ ($i=3, 4, 5, 6$) is distinct from $G_1$ (thickness and aspect ratio). 
           \end{itemize}
       \end{itemize}
       
           

    \subsubsection{Purpose: Assumption Verification}
      \par
       Without calculating the asymptotic expansion, the survival function can characterize and differentiate branching objects.
    


  \subsection{LRWs Simulation}

     \subsubsection{Survival Functions}

        \subsubsubsection{$S(n)$}
            \par
            $n$ is the number of steps taken by the particle.
        \subsubsubsection{$S(d)$}
            \par
            $d$ is the displacement from the initial to the final position of a particle undergoing LRWs.
        \subsubsubsection{Empty Space Function: $S(R)$}
            \par
            $R$ is the radius of the largest disk, which is centered at the initial position and tangent to the root system \cite{baddeley1994empty}.

     \subsubsection{Explore Information From Survival Curves}
            
     \subsubsection{Output Analysis}
        \begin{itemize}
          \item Two-sample statistical tests of $S(n)$
          \item Distance matrices
          \item Multidimensional scaling plot  
        \end{itemize}
        

          
         
     
      

  \subsection{Conclusion}
    \begin{itemize}
      \item In a short time, the survival function of rectangle decays faster than the circle, which conforms to the analytical results.
  
      \item The differences of estimated survival functions between circle and rectangle are statistically significant, which coincides with the real shape dissimilarities.

      \item Within a same group, when $t$ is small, the more branching the object is, the faster the survival function decays.

      \item Within a same group, the pairwise survival functions are statistically different.

      \item The corresponding target structures in $G_1$ and $G_3$ are invariant shapes under translation since their survival function are not statistically different. In other words, periodic boundary conditions of the image can eliminate the effect of the locations.

      \item LRWs can describe and classify the geometries, their spatial configurations, and the unoccupied area in the image.
    \end{itemize}
    
    




