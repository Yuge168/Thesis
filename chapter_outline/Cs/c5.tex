
  \subsection{Hypothesis}
   \par
   Given $2-$dimensional root images, LRWs can be used to characterize and compare the plant root shapes at all scales.
  
  \subsection{Image Description}
   
     \subsubsection{Background}
     \subsubsection{$2-$dimensional Sorghum Images}
       \begin{itemize}
         \item Genotypes
         \item Treatments
         \item Replicates
       \end{itemize}
       
    
  \subsection{Output Analysis}
     \subsubsection{Survival Functions}
       \begin{itemize}
         \item $S(n)$, where $n$ is the number of steps taken by the particle.
         \item $S(d)$, where $d$ is the displacement from the initial to the final position of a particle undergoing LRWs.
         \item $S(R)$, where $R$ is the radius of the largest disk, which is centered at the initial position and tangent to the root system.
       \end{itemize}
       
      
     \subsubsection{Analyze Survival Curves}
        \begin{itemize}  
          \item Explore the connection between particles' behaviours and the shape of the survival curve.
          \item Describe morphological features and spatial distribution of roots.
          \item Characterize the unoccupied area of the image
        \end{itemize}
        

    \subsubsection{Cox Proportional Hazards Model}
       \subsubsubsection{Purpose}
         \par
          Evaluate simultaneously the effects of several factors on survival. Or examine how specified factors influence the hazard rate - the rate of a particular event happening at a particular point in time.  
      
    \subsubsection{Fr\'echet Mean and Variance}

    
    \subsubsection{Distance Matrics}
      \begin{itemize}
        \item Survival Function
        \item Fr\'echet Mean
      \end{itemize}
      
     
  \subsection{Conclusion}


