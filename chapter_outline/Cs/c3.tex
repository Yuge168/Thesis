\begin{outline}[enumerate]

  

\1 Analytical Results
    \2 Shape Description
      \3 Problem domain $\Omega$: the region bounded by two concentric circles
      \3 Radius of the larger circle: $b$
      \3 Radius of the smaller circle: $a$
    \2 Solving Initial-Boundary Value Problem (IBVP)
      \3 Methods
        \4 Dimensional Analysis: non-dimensional variables
          \begin{itemize}
            \item $\mu = \frac{b}{a}$
            \item $\tau = \frac{t}{a^2}$
            \item $\hat r = \frac{r}{a}$
          \end{itemize}
        \4 Method of separation of variables
     \3 Mathematical Equations
       \4 Diffusion equation
       \begin{equation}\label{eq:DA_polar_diffusion}
         u_\tau = (u_{\hat r \hat r} + \frac{1}{\hat r} u_{\hat r} + \frac{1}{\hat r ^2} u_{\theta\theta})
       \end{equation}
       \4 Uniform initial condition
       \begin{equation}\label{eq:DA_initial_bc}
         u(\hat r, \theta, 0) = \frac{1}{|\Omega|}
       \end{equation}
       \4 Homogenous Dirichlet B.C.
       \begin{equation}\label{eq:DA_Dirichlet_bc}
         u(1, \theta, \tau) = 0
       \end{equation}
       \4 Homogenous Neumann B.C.
       \begin{equation}\label{eq:DA_Neumann_bc}
         \hat r u'(\mu, \theta, \tau) = 0
       \end{equation}
       
    \2 Heat Content Calculation
      \begin{equation}\label{eq:heat_content_annulus}
        S(\tau) = \int_{0}^{2\pi} d\theta \int_{1}^{\mu} \hat r d \hat r u(\hat r, \theta, \tau)
      \end{equation}
    
  \1 Numerical Approximation
    \2 Eigenvalues $\lambda_{0, n}$
      \3 Properties
        \4 $\lambda_{0, n} \in \mathbb{R^{+}}$, $(n \in \mathbb{N}_{+})$
        \4 Monotonicity 
        \4 Periodicity
      \3 Estimation
        \4 $\lambda_{0, n} \in ((n-1) \pi, (n+1) \pi)$ \cite{NIST:DLMF}
        \4 Bisection method \cite{2020SciPy-NMeth}

    \2 Approximation of $u(\hat r, \theta, \tau)$ and $S(\tau)$
      \3 Direct summation
      \3 Series acceleration methods   

  \1 LRWs in Annulus
    \2 Model Design
      \3 Uniform Sampling
      \3 B.C.s
    \2 Survival analysis in Python \cite{cameron_davidson_pilon_2020_4313838}
       \3 Survival function $S(n)$, where $n$ is the number of steps taken by the particle. 
       \3 Confidence interval
    \2 Sampling Errors

  \1 Comparison of Numerical and Analytical Results
    \2 Sample Size Evaluation
    \2 Comparison of $S(\tau)$ and $S(n)$

  \1 Conclusion
    \2 The estimated survival function of LRWs is consistent with the analytical result.
    \2 The number of particles in LRWs determined by DKW inequality is large enough to generate reproducible statistical results. 

      





\end{outline}
