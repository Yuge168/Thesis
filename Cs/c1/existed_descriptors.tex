


\section{Summary of Existed Descriptors}
 
   \subsection{Metric}

     \subsubsection{Basic Geometric Descriptors}
       \begin{itemize}
         \item maximum depth
         \item maximum width
         \item etc.
       \end{itemize}
       
     \subsubsection{Compound Descriptors (Computed From the Basic Descriptors)}
       \begin{itemize}
         \item Density
         \item aspect ratio
         \item etc.
       \end{itemize}
       
     \subsubsection{Weaknesses}

       \begin{itemize}
         \item Rely on the resolution of the images \cite{balduzzi2017reshaping}
         \item Only provide a general view of root morphology \cite{balduzzi2017reshaping}
         \item Difficult to assess the spatial configuration of roots
         \item Fail to describe the full complexity of root systems
       \end{itemize}

       
   \subsection{Non-metric}
       
    \subsubsection{Topological Analysis}

      \begin{itemize}
        \item Persistent homology \cite{li2017persistent}
          \begin{itemize}
            \item Persistence barcode shows the number of branches.
            \item Persistence barcode indicates how branched roots connect along the scale of the function (e.g. geodesic distance).
            \item Compare the similarity of branching structures by a pair-wise distance matrix using the bottleneck distance method.
          \end{itemize}
          
       \item Horton-Strahler index \cite{toroczkai2001topological}
         \begin{itemize}
            \item Categorize the topological complexity of the whole branching structure.
            \item Provide a numerical measure of connectedness and complexity of the branching at each vertex by a dimensionless ratio: bifurcation ratio.
            \item The range of index and the length ratio indicate the size of the branching structure.
         \end{itemize}
                
       \item Fractal Analysis \cite{tatsumi1989fractal}
         \begin{itemize}
            \item Measure the complexity of branching structures.
            \item Measure self-similarity of branching structures by fractal dimension of the root systems.
         \end{itemize}
      \end{itemize}
      

   \subsubsection{Strengths}
     \begin{itemize}
       \item Highly complementary to geometric descriptors to characterize how individual roots are connected through branching \cite{delory2018archidart}.
       \item Describe branching structures independent of transformation and deformation.
     \end{itemize}
     

   \subsubsection{Problems and Weaknesses}
     \begin{itemize}  
       \item Only analyze the connectedness of branching structures, which is a portion of the complexity of plant root systems.
       \item Fail to characterize spatial distribution.
       \item Some biologically topological indices analyze the root growth qualitatively based on line-linked systems \cite{fitter1986topology}, but not characterize the mathematical topological properties.
       \item The fractal analysis aims to describe self-similar structures, which grow by continually repeating simple growth rules \cite{fitter1992fractal}.
     \end{itemize}
     

       
