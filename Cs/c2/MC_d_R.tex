

\section{Novel Applications of Monte Carlo Simulation for Shape Characterization}

Survival analysis is a series of statistical methods dealing with
variables that have both a time and event associated with them. It can
be applied in a variety of fields \cite{clark2003survival}, including
cancer studies in clinical research, event-history analysis in
sociology, failure-time analysis in engineering, and so on. As
introduced in the last section ~\ref{section:MC_heat_content}, the
event we are interested in is whether the particle hit the absorbing
boundary or not, and the time refers to the number of steps, $n$,
taken by the particle in LRWs. However, in this section, two novel
forms of functions stem from the underlying survival process are
developed for characterizing the shape of the object in the image.


\subsection{Survival Analysis of Displacement $S(d)$}

In some survival analysis applications, the origin-destination
distance can be an alternative random variable to the time. For
example, in the study of transport habits
\cite{anastasopoulos2017transport} \cite{su12166331}, trip
distance-based survival analysis can reveal traveller preference and
trip patterns and measure the effects of explanatory variables. In
Monte Carlo simulations, the displacement of a particle, $d$, is the
shortest distance from the initial to the stop position in the
infinite tiling space, which is affected by the number of steps
$n$. Suppose $\mathcal{D}$ is a non-negative random variable
representing the displacement moved by the particle from the origin to
any sites of absorbing boundary. Like the $S(n)$, for any $d>0$, we
can define $S(d)$ as

\begin{equation}\label{eq:sf_disp}
  S(d) = Pr(\mathcal{D} > d) = 1 - F(d)
\end{equation}
where $S(d)$ indicates the probability that the particle has not yet
been absorbed when the displacement beyond $d$, and $F$ is the
cumulative distribution function. Moreover, both $S(d)$ and $F(d)$ are
right continuous in this thesis.

As mentioned before, the KM \cite{kaplan1958nonparametric}, or
product-limit, can be used to estimate the survival function
nonparametrically. Suppose that $N$ particles determined by
Dvoretzky–Kiefer–Wolfowitz (DKW) inequality
\cite{dvoretzky1956asymptotic} undergo LRWs independently in the
simulation with distinct increasing non-negative displacements $d_1<
d_2< ...<d_j <...<d_k$. The probability of particles being alive at
displacement $d_j$, $S(d_j)$, is calculated from $S(d_{j-1})$ by

\begin{equation}\label{eq:km_disp}
  S(d_j) = S(d_{j-1}) \Big( 1-\frac{\tilde{n}_j}{n_j} \Big)
\end{equation}
where $n_j$ is the number of particles have not hit the target
boundary within the displacement $d_j$, and $\tilde{n}_j$ is the number of
particles being absorbed at displacement $d_j$. 




\subsection{Survival Analysis of Radius S(r)}


Another function generated in this thesis for analyzing the spatial
point patterns in the image is based on a union of discs of
independent random radius centred at the uniform sampling points. The
radius, $R$, of a disc is the shortest distance from an arbitrary
initial site, $s \in \bm{I} \subseteq \mathbb{R}^2$, to the fixed
absorbing boundary, $\Gamma \subseteq \mathbb{R}^2$, in the whole
tiling of the plane. $R$ is defined as

\begin{equation}\label{eq:dist_point_set}
  R = \rho(\Gamma, s) = \inf \left\{ \lVert a-s \lVert : a \in \Gamma \right\}
\end{equation}


The empty space function \cite{baddeley2007spatial} of a point process
\cite{zahle1982random} gives us an insight into the distribution of
the random variable $R$. A point process is a collection of points
located randomly in some underlying mathematical spaces. It evolved
gradually from renewal theory and the statistical analysis of life
tables dating back to the $17$th century
\cite{daley2007introduction}. Point process data has wide applications
in various fields \cite{daley2007introduction}, including
epidemiology, ecology, forestry, mining, astronomy, ecology,
meteorology, etc.

$1-$dimensional point process can be used to model a sequence of
random times when a specific event happens. For example, given any
period of time, a call center can receive calls at random instants or
points of time. Suppose we map the caller locations on a particular
day. Then, the map will constitute a random pattern of points in two
dimensions because there will have an arbitrary number of such points,
and their positions are random. Therefore, a spatial point process can
model the random pattern of points in $2-$dimensional space. If both
the time and locations are recorded, we can consider it as a
space-time point process in three dimensions. It is also a simple
instance of the marked point process \cite{baddeley2007spatial} since
each location in $\mathbb{R}^2$ is labelled or marked by the time of
the call in $\mathbb{R}$.

Let $\bm{X}$ be a stationary point process in $\mathbb{R}^k$, the
empty space function \cite{baddeley2007spatial}, $F(r)$, also named
spherical contact distribution function, is the cumulative
distribution function of the contact distance, $dist(u, \bm{X})$, from
a fixed point $u$ to the nearest point of random set $\bm{X}$.

\begin{equation}\label{eq:empty_space_function}
  F(r) = Pr(dist(u, \bm{X}) \leq r) = Pr(N(b(u, r)) > 0)
\end{equation}
where $r \geq 0$, and $N(b(u, r))$ is a random variable indicating the
number of discs with radius $r$ centered at $u$. $F(r)$ is independent
on $u$ because of the stationarity.


$$......... how ? why ?$$


%Since all points, initial positions of particles, are uniformly and
%independently placed in the empty space in the image

In the light of the definition of empty space function of a stationary
or uniform point process, the survival function $S(r)$ can be
expressed as

\begin{equation}\label{eq:sf_radius}
  S(r) = Pr(R > r) = 1 - F(r) = 1 - Pr(R \leq r)
\end{equation}




