

\section{Kac’s Idea: Can One Hear the Shape of a Drum? \cite{kac1966can}}



  \subsection{Interpretations of Kac's Problem}

     \begin{itemize}
       \item When the drum vibrates, one can hear the sound, which is composed of tones of various frequencies. How much can shape features be inferred from hearing a discrete spectrum of pure tones produced by a drum?
       \item If a complete sequence of eigenvalues of the Dirichlet problem for the Laplacian can be obtained precisely, will people determine the shape of a planar?
     \end{itemize}




    \subsection{Problem Statement}

     \begin{itemize}     
       \item Consider a simply connected membrane $\Omega$ in the Euclidean space bounded by a smooth convex curve $\partial \Omega$ (e.g. a drum without any holes)
       \item Find function $\phi$ on the closure of $\Omega$, which vanishes at the boundary $\partial \Omega$, and a number $\lambda$ satisfying $-\Delta \phi = \lambda \phi$.

         \begin{itemize}
           \item $\Delta$ is the Laplace operator. e.g. $\Delta = \sum_{i=1}^{n} \frac{\partial ^2}{\partial x_i^2}$ in Cartesian coordinate system.
           \item If there exists a solution $\phi \neq 0$, the corresponding $\lambda$ is defined as a Dirichlet eigenvalue.
           \item For each domain $\Omega$, there has a sequence of eigenvalues $\lambda_1, \lambda_2, \lambda_3, ... $ corresponding to a set of eigenfunction $\phi_1, \phi_2, \phi_3, ...$.
           \item $\phi_k$ form an orthonormal basis of $L^2(\Omega)$ of real valued eigenfunctions; the corresponding discrete Dirichlet eigenvalues are positive ($\lambda_k \in \mathbb{R}^{+}$).
         \end{itemize}
         
       \item An important function \cite{grieser2013hearing}:

         \begin{equation}\label{eq:heat_trace}
           h(t) = \sum_{k=1}^{\infty} e^{-\lambda_kt}
         \end{equation}
         
         \begin{itemize}  
           \item It is a Dirichlet series.
           \item It is called the spectral function or the heat trace.
           \item It is smooth and converges for every $t>0$.
         \end{itemize}
     \end{itemize}



     \subsection{Summarize the Results of Kac's Idea}

       \begin{equation}\label{eq:kac_result}
          h(t) = \sum_{k=1}^{\infty} e^{-\lambda_kt} \sim \frac{|\Omega|}{2\pi t} - \frac{L}{4} \frac{1}{\sqrt{2\pi t}} + \frac{1}{6}
       \end{equation}
     
      \begin{itemize} 
        \item As $t \rightarrow 0^{+}$, the leading terms of the asymptotic expansion of $h(t)$ imply the geometrical attributes of $\Omega$
          \begin{itemize}
            \item the total area
            \item the perimeter
            \item the curvature
          \end{itemize}
          
        \item If the domain $\Omega$ has the polygonal boundary, the third term shows in the information about the interior angles of the polygon \cite{grieser2013hearing}.
      \end{itemize}



    \subsection{Conclusion}

     \subsubsection{Advantages}

        \begin{itemize}
          \item Kac proposed a novel analytical mathematical method for the shape description without using measuring tools, e.g. rulers.
          \item Other mathematicians extended Kac's idea in exploring the geometrical information of more complex domains with various boundary conditions \cite{khabou2007shape}\cite{gottlieb1985eigenvalues}\cite{gottlieb1983hearing} \cite{zayed1989heat}\cite{sleeman1984trace}.
        \end{itemize}
        
     \subsubsection{Limitations}
        
        \begin{itemize}
          \item It is only available for the convex domain, which has a smooth or piecewise smooth boundary.  
          \item Except in very few cases (i.e. rectangular, disk, certain triangles), the complete sequence of eigenvalues $\lambda_k$ can not be calculated \cite{grieser2013hearing}.
          \item Only the first few terms in the asymptotic expansion of $h(t)$ are explicitly available.
        \end{itemize}
