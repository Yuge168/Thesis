\section{Extended Work of Kac’s Idea \cite{desjardins1994heat}\cite{vandenberg1994heat}: Heat Content}


\subsection{Fouier's Heat Equation \cite{baron1878analytical}}

  \begin{itemize}
    \item Mathematical Formula
        \par
        \begin{equation}\label{eq:heat_equation}
          \frac{\partial u(\bm{s}, t)}{\partial t} = \Delta u(\bm{s}, t)
        \end{equation}

   \item Interpretation
     \par
     It is a deterministic model used to characterize the evolution of quantities over the space and time. (e.g. the flow of heat)
  \end{itemize}




\subsection{Summarize the Idea}

   \begin{itemize}
      \item Initial-Boundary Value Problem (IBVP)
        \par
        $u(\bm{s}, t)$ indicates the value of the tempretature at $\bm{s} \in \Omega$ at time $t$ satisfying Eq.~\ref{eq:heat_equation} and
        \begin{itemize}
          \item Initial condition: $u(\bm{s}, t) = f(\bm{s})$ as $t \rightarrow 0$.
          \item Dirichlet boundary condition: $u(\bm{s}, t)=0$ for $\bm{s} \in \partial \Omega$
            \par
            It is also called the absorbing boundary condition; i.e. any molecule will be instantly absorbed when it touches the boundary $\partial \Omega$;
        \end{itemize}

      \item A Basic Integration

        \par
          \begin{align}
            \beta_{\Omega}(f, g)(t) &= \int_{\Omega} \int_{\Omega} H_{\Omega}(\bm{s}, t | \bm{s_0}) f(\bm{s_0}) g(\bm{s}) d\bm{s_0} d\bm{s} \label{eq:integral_full} \\
            &= \int_{\Omega} u(\bm{s}, t) g(\bm{s}) d\bm{s} \label{eq:integral_convol}
          \end{align}


        \begin{itemize}
           \item $H_{\Omega}(\bm{s}, t | \bm{s_0})$ is called the heat kernel of $\Omega$ describing the density of the heat at $\bm{s}$ after time $t$ when initially there is only one single hot source at $\bm{s_0}$.
           \item $u(\bm{s}, t)$ is the general solution to Eq.~\ref{eq:heat_equation}, which can be expressed as the convolution of the initial condition with the heat kernel of the domain.
           \item $g(\bm{s})$ is an auxiliary test function for studying the distributional nature of the tempretature function $u(\bm{s}, t)$ near $\partial \Omega$.
        \end{itemize}


      \item Heat Content Calculation

        \par
       Given 
       \begin{equation} \label{eq:g}
         g(\bm{s}) = 1  
       \end{equation}

       \begin{align}
         Q_{\Omega}(t) &= \int_{\Omega} \int_{\Omega} H_{\Omega}(\bm{s}, t | \bm{s_0}) f(\bm{s_0})  d\bm{s_0} d\bm{s} \label{eq:heat_content_integral_full} \\
            &= \int_{\Omega} u(\bm{s}, t) d\bm{s} \label{eq:heat_content_integral_convol}
       \end{align}


     \item Shape Characterization

       \begin{itemize}
       \item As $t \rightarrow 0^{+}$,

         \begin{equation}\label{eq:heat_content_asy}
           Q_{\Omega}(t) \simeq \sum_{n=1}^{\infty} \beta_n(\Omega) t ^{\frac{n}{2}}
         \end{equation}
         
         
         \item Obtain geometrical information of $\Omega$ from $\beta_n$
           \begin{itemize}
           \item area
           \item length
           \item scalar curvature
           \item mass
         \end{itemize}
       \end{itemize}
       
       
   \end{itemize}



  \subsection{Conclusion}

    \begin{itemize}
      \item Strengthness
        \begin{itemize}
          \item Instead of calculating a complete sequence of the Dirichlet eigenvalues for exploring the shape attributes of geometry, the asymptotic expansion of the heat content, defined as integrating the solution to the heat equation over the space-dimension, also implies the geometrical characteristics.  
        \end{itemize}
        
      \item
        \begin{itemize}
          \item Only the infinitly differentiable boundary $\partial \Omega$ is considered.
          \item Only the first few terms in the asymptotic expansion are explicitly known.
          \item Either irregular geometries or discontinuities lead to the complexities, so the explicit solutions $u(\bm{s}, t)$ are close to non-existed.
          \item The numerical evaluation of the analytical $u(\bm{s}, t)$ and $Q_{\Omega}(t)$ is usually by no means trivial because they are in the form of infinite series.
          \item Similiarly, only the first few coeffients $\beta_n$ in the asymptotic expansion of $Q_{\Omega}(t)$ can be expressed as the complicated explicit forms.
        \end{itemize}
        
    \end{itemize}

    
  
    
   
   




