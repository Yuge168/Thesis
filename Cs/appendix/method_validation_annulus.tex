\label{appendix:method_validation_annulus}

\section{Analytical Results}

   \subsection{Shape Description}
     \begin{itemize}
      \item Problem domain $\Omega$: the region bounded by two concentric circles
      \item Radius of the larger circle: $b$
      \item Radius of the smaller circle: $a$
     \end{itemize}
     
   \subsection{Solving Initial-Boundary Value Problem (IBVP)}
     
     \subsubsection{Methods}

       \begin{itemize}
         \item  Dimensional Analysis: non-dimensional variables
            \begin{itemize}
              \item $\mu = \frac{b}{a}$
              \item $\tau = \frac{t}{a^2}$
              \item $\hat r = \frac{r}{a}$
            \end{itemize}
            
          \item Method of separation of variables
       \end{itemize}

    \subsubsection{Mathematical Equations}
      \begin{itemize}
        \item Diffusion equation
          \begin{equation}\label{eq:DA_polar_diffusion}
             u_\tau = (u_{\hat r \hat r} + \frac{1}{\hat r} u_{\hat r} + \frac{1}{\hat r ^2} u_{\theta\theta})
          \end{equation}

        \item Uniform initial condition
          \begin{equation}\label{eq:DA_initial_bc}
            u(\hat r, \theta, 0) = \frac{1}{|\Omega|}
          \end{equation}

        \item Homogenous Dirichlet B.C.
          \begin{equation}\label{eq:DA_Dirichlet_bc}
            u(1, \theta, \tau) = 0
          \end{equation}

        \item Homogenous Neumann B.C.
           \begin{equation}\label{eq:DA_Neumann_bc}
              \hat r u'(\mu, \theta, \tau) = 0
           \end{equation}
           
      \end{itemize}

       

     \subsubsection{Heat Content Calculation}
       \begin{equation}\label{eq:heat_content_annulus}
         S(\tau) = \int_{0}^{2\pi} d\theta \int_{1}^{\mu} \hat r d \hat r u(\hat r, \theta, \tau)
       \end{equation}
       
          
\section{Numerical Approximation}

   \subsection{Eigenvalues $\lambda_{0, n}$}
     \begin{itemize}
       \item Properties
         \begin{itemize}
            \item $\lambda_{0, n} \in \mathbb{R}^{+}$, $(n \in \mathbb{N}_{+})$
            \item Monotonicity and Periodicity
         \end{itemize}
         
       \item Estimation
         \begin{itemize}
          \item $\lambda_{0, n} \in ((n-1) \pi, (n+1) \pi)$ \cite{NIST:DLMF}
          \item Bisection method \cite{2020SciPy-NMeth}
         \end{itemize}
   
     \end{itemize}

 
   \subsection{Approximation of $u(\hat r, \theta, \tau)$ and $S(\tau)$}
     \begin{itemize}
       \item Direct summation
       \item Series acceleration methods
     \end{itemize}
     

 \section{Comparison of Numerical and Analytical Results}

   \subsection{Sample Size Evaluation}

   \subsection{Comparison of $S(\tau)$ and $S(n)$}


   
\section{Conclusion}

  \begin{itemize}
    \item The estimated survival function of LRWs is consistent with the analytical result.
    \item The number of particles in LRWs determined by DKW inequality is large enough to generate reproducible statistical results. 
  \end{itemize}

