
    \section{Conclusion}\label{section:branch_conclusion}

      \begin{itemize}
         \item In a short time, the survival function of rectangle decays faster than the circle, which conforms to the analytical results.
  
         \item The differences of estimated survival functions between circle and rectangle are statistically significant, which coincides with the real shape dissimilarities.

         \item Within a same group, when $t$ is small, the more branching the object is, the faster the survival function decays.

         \item Within a same group, the pairwise survival functions are statistically different.

         \item The corresponding target structures in $G_1$ and $G_3$ are invariant shapes under translation since their survival function are not statistically different. In other words, periodic boundary conditions of the image can eliminate the effect of the locations.

         \item LRWs can describe and classify the geometries, their spatial configurations, and the unoccupied area in the image.
    \end{itemize}
