\section{Complicated Branching Structures}


In the preceding section, LRWs are simulated in the $2-$dimensional
images of equal-area simply connected domains (i.e. without holes in
them) with smooth or piecewise smooth boundary. Both short and
long-term behaviours of survival functions for the circle and
rectangle are analyzed and compared to verify LRWs in the application
of shape comparison. However, the plant root shape is an extremely
complicated branching structure, consisting of the primary roots,
taproots, secondary roots, etc. 

Hence, this section is outlined as follows. First of all, some
branching structures, as shown in Fig.~\ref{fig:G1_imgs} and
Fig.~\ref{fig:G2_imgs}, are designed and generated in equal size
$2-$dimensional images. Secondly, without the simulation of LRWs, the
spatial pattern and structural features of the branching structures
are discussed by the interpretation of the survival function $S(R)$,
where $R$ is the nearest distance from an arbitrary point in the space
to the fixed boundary of the target object. Thirdly, the other two
survival functions, $S(n)$ and $S(d)$, of the branching structures are
estimated and compared, where $n$ and $d$ are the numbers of steps
taken by the particle from the initial to the stop position and its
corresponding displacement in the tiling space, respectively. However,
three types of survival functions and their interpretation of the
underlying stochastic process in this section are intimately related.



\subsection{Artificial Branching Structure Construction}


The plant root shape can be represented as a spatial structural system
consisting of separate branches, and each of them forks out at a
specific point, or knot, into at least two branches.  Moreover, such
branching structure can constantly subjected to tension, compression,
flexion, bending, and torsion.  In this section, as shown in
Fig.~\ref{fig:G1_imgs} and Fig.~\ref{fig:G2_imgs}, the design of two
groups of artificial images builds upon and extends the definition of
the perfect binary tree structure \cite{rosen1999discrete} in graph
theory \cite{west2001introduction}. A binary tree is a data structure
consisting of nodes, each of which has up to two childer named left
and right child nodes. It starts with a single topmost node called the
root. A perfect binary tree is a specific type of binary tree in which
all the interval nodes are connected with two child nodes, and all
leaves are at the same level. The root is at level $0$, its child
nodes are at level $1$, and so on.



    \begin{figure}
        \centering
        \begin{subfigure}[b]{0.45\textwidth}
          \includegraphics[width=\textwidth]{G_1_L_1.png}
          \caption{}
          \label{fig:g1_template}
        \end{subfigure}
        \hfill
        \begin{subfigure}[b]{0.45\textwidth}
          \includegraphics[width=\textwidth]{G_2_L_1.png}
          \caption{}
          \label{fig:sf_g2_template}
        \end{subfigure}
        \caption{(a) and (b) are branch templates with identical area
          for generating artificial branching structures in $G_1$ and
          $G_2$, respectively. Templates in (a) and (b) have same
          maximum width (i.e. $150$), but different maximum depth
          (i.e. $1050$ and $1200$, respectively).}
        \label{fig:branch_templates}
    \end{figure}
    



As shown in Fig.~\ref{fig:branch_templates}, the branch templates for
$G_1$ and $G_2$ are slightly distinct in shapes, connection, the
thickness of the left and right child branches, and aspect
ratio. According to the concept of perfect binary tree structures, the
template can be decomposed into three branches, one main or parent
branch and two child branches. The main branch is connected with a
left and right child branch equidistantly and symmetrically without
voids and overlays. The equal-area artificial branching structures in
Fig.~\ref{fig:G1_imgs} and Fig.~\ref{fig:G2_imgs} are generated based
on the principle of fractal geometry \cite{falconer2004fractal} by
repeating the branch template at an increasingly smaller scale from
top to bottom. The following algorithm summarizes how to construct a
binary branching structure:

  \begin{algorithm}
    \SetAlgoLined
    \KwResult{$2-$dimensional Artifical Branching Image with Area $A$}

      Set number of levels of the branching structure $j$\;
      Given a branch template $\widehat{T}$ with a predefined surface area $A$\;
      Generate an all-black image $Img$\;
      
      \For{$l = 0;\ l < j;\ l = l+1$}{
        
          define $l-$th level branch template as $\widehat{T}_{j, l}$ by multiplying $\widehat{T}$ with a scaling factor $s_{j,l}$\;
          
          \eIf{$l == 0$}{

            draw a branch template $\widehat{T}_{j, l}$ in $Img$\;
            find the bottom left and right convex corners of $\widehat{T}_{j, l}$\;
          }{

            draw and put $2^l$ branch templates, $\widehat{T}_{j, l}$, at every convex corner\;
            find the bottom left and right convex corners of all $\widehat{T}_{j, l}$\;
            take a union of previous $Img$ with the collection of smaller-scale branches and redefine $Img$\;   

          }        
      }
      
      \eIf{number of white pixels $N$ $!=$ A}{
        add $A-N$ white pixels around branch templates in $Img$
      }
    
      \caption{Generate Artifical Branching Structure}
      \label{alg:branch_iteration}
  \end{algorithm}


Those artificial structures in Fig.~\ref{fig:G1_imgs} and
Fig.~\ref{fig:G2_imgs} are iteratively-defined geometries as described
in the algorithm \ref{alg:branch_iteratio}, and their complexities
grow with the number of hierarchical branching or levels $j$, where
$j=3, 4, 5, 6$. The large-scale or low-level branches mimic thick
roots near the soil surface. In contrast, the smaller-scale branches
at the higher level represent the fine roots that gradually become
vertical in deeper horizons because of the positive response of
gravity in soil.


From the perspective of mathematics, in each image, the region
occupied by white pixels and bounded by a piecewise smooth curve is a
$2-$dimensional simply connected domain. As similar to the methodology
validation in the last section, we will compare the short-term
survival function behaviours and assess their discrepancy by
statistical results. However, the primary purpose is to interpret the
survival curves and reveal the spatial pattern information.









      

      
    






