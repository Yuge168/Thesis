
The fixed-time step Monte Carlo simulation, LRWs, has been validated
in the annulus in Appendix \ref{appendix:method_validation_annulus} by
comparing the analytical heat content and numerical survival
function. The further validation of LRWs is based on
Eq.\ref{eq:heat_content_asy}, which indicates that the geometric
features of a region with a smooth boundary in $\mathbb{R}^2$ can be
deduced from the asymptotic properties of the heat
content. Specifically, the coefficient of the first term is only
associated with the total area of the region, and the second one can
only be affected by the total length of the boundary of the
region. Theoretically, for the equal-area shapes, the larger perimeter
causes a faster decay rate of the short-term survival function because
of a minus sign in front of the second term in the asymptotic series.


However, unlike simulating LRWs inside of the target region, i.e. the
annulus in Appendix \ref{appendix:method_validation_annulus},
particles are distributed uniformly outside of the target shape filled
with white pixels and undergo LRWs in the unoccupied space filled with
black pixels in the $2-$dimensional image. In other words, the
geometric attributes and spatial patterns of the target shapes are
reflected and implied through exploring the unoccupied space by
particles in the image. Therefore, we are interested in acknowledging
whether the short-term behaviours of survival functions of the
numerical simulation still coincide with the analytical
results. Moreover, it is vital to assess the difference between
survival functions to test whether they are reliable to compare and
distinguish the shapes. 


In this chapter, the preliminary step of testing the hypothesis is to
generate black-and-white images with the same dimensions as shown in
Fig.~\ref{fig:simple_imgs}, Fig.~\ref{fig:G1_imgs}, and
Fig.~\ref{fig:G2_imgs}. In the binary images, the objects are convex
shapes with the same area or an equal number of white pixels. The next
step is to simulate LRWs and estimate survival functions by
Kaplan-Meier estimator \cite{kaplan1958nonparametric} implemented in
lifelines library in Python
\cite{cameron_davidson_pilon_2021_4505728}. Then, some commonly
applied nonparametric statistical tests, as proposed in section
~\ref{section:MC_heat_content}, are utilized to test the equality of
survival functions. Except for investigating how survival functions
differ from each other by statistical tests and dissimilarities
calculation, and it is meaningful and helpful to reveal why they are
distinct by studying the underlying stochastic process.
