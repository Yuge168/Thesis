


Some commonly used descriptors for plant root morphology in $2-$
dimensional black-and-white images and their limitations are
summarized in chapter ~\ref{chap:existing_descriptors}. Overall, some
of them only extract and characterize a set of representative
geometric features of plant roots from images, and some can only
describe the connectedness of branches. However, none of them provide
information about the spatial distribution of roots.


In chapter \ref{chap:math_methods}, we have a perceptual shift and
begin understanding the plant root images from a reversing view, that
is, to explore and distinguish the unoccupied space in the image,
i.e. a region without roots. Also, we are intrigued by Kac's
idea \cite{kac1966can} about using analytical mathematical tools for
shape characterization to overcome the limitations of conventional
ruler-based measurements. Based on the research inspariation, two
mathematical processes, point process and stochastic process
(i.e. LRWs), are proposed to explore and characterize the unoccupied
space and summarize the geometric attributes and spatial arrangements
of object in the image without rulers.



From the point process \cite{zahle1982random} view, for a
$2-$dimensional image, we can map the object or figure consisting of
white pixels as a fixed set of points and regard the uniform
distributed points in the unoccupied space as a $2-$dimensional random
point pattern. The survival function of radius or shortest distance,
$R$, from the fixed set to random points of a point process can be
estimated by Kaplan-Meier
estimator \cite{kaplan1958nonparametric} \cite{aalen2008survival} \cite{cameron_davidson_pilon_2021_4505728}. Therefore,
one of the primary purposes of this chapter is to interpret the
survival function of radius, $S(R)$, and to demonstrate the underlying
spatial information about an image.



The heat content
calculation \cite{desjardins1994heat} \cite{vandenberg1994heat} is an
extension of Kac's idea \cite{kac1966can}, which presents a relatively
practicable mathematical technique to describe an irregular compact
region with a smooth boundary. Eq.\ref{eq:heat_content_asy} indicates
that the geometric features of a region with a smooth boundary in
$\mathbb{R}^2$ can be deduced from the asymptotic properties of the
heat content, which is the integration of the solution to the heat
equation over the predefined domain. However, the implementation in
practice is by no means trivial. First of all, the analytical methods
and solutions are restricted to simple geometries and constant
diffusion properties, but the image unoccupied region is usually
extremely complicated. Additionally, the numerical evaluation of the
solution, heat content, and asymptotic expansion is complex or
sometimes inaccessible because they are in the form of infinite
series. Last but not least, more spatial information might be hidden
in the long-term behaviours of heat content. Therefore, a stochastic
process \cite{schuss2009theory}, i.e. lattice random walks (LRWs), is
proposed, and the fixed-time step Monte Carlo simulation is designed
to approximate the heat content.


The fixed-time step Monte Carlo simulation, LRWs, has been validated
in the annulus in Appendix \ref{appendix:method_validation_annulus} by
comparing the analytical heat content and numerical survival
function. In a word, the estimated survival function of $N$ particles
undergoing LRWs in the annulus converges to the analytical heat
content, where $N$ is determined by DKW inequality in section
~\ref{section:sample_size_determination}. Eq.\ref{eq:heat_content_asy}
gives a clue on how to carry out the further validation of LRWs. The
coefficient of the first term is only associated with the total area
of the region, and the second one can only be affected by the total
length of the boundary. Hence, theoretically, for the equal-area
shapes, the larger perimeter causes a faster decay rate of the
short-term survival function because of a minus sign in front of the
second term in the asymptotic series in Eq.\ref{eq:heat_content_asy}. 


The fundamental motivation behind equal-area branching structures
devised in this chapter is the predictable short-term survival
function behaviours in Eq.\ref{eq:heat_content_asy}. Without
calculating the asymptotic expansion, we are interested in knowing
whether short-term behaviours of survival functions coincide with
analytical results, testing whether survival curves for distinct
shapes are statistically distinguishable, and understanding the
survival curve to unveil the geometric and spatial information about
the image.  


 
Thus, this chapter is outlined as follows. In section
~\ref{section:circle_rectangle}, two simple regular shapes with the
identical area, i.e. the circle and rectangle, are generated in
black-and-white images with the same dimensions as shown in
Fig.~\ref{fig:simple_imgs}. Commonly applied two-sample statistical
tests assess the distinctions between the survival functions of the
circle and the rectangle, which is a preliminary examination of LRWs
to know whether we can differentiate geometries by comparing survival
curves. In section ~\ref{section:branching_structures}, the
motivation, layout, and algorithm of producing the branching
structures shown in Fig.~\ref{fig:G1_imgs} and Fig.~\ref{fig:G2_imgs}
are construed in detail. And then, the survival functions of the
radius (i.e. $R$), the number of steps (i.e. $n$), and the distance
(i.e. $d$) are interpreted and analyzed by exploring the underlying
static and dynamic random processes, which reveal both geometric and
spatial attributes of branching structures. Except for studying why
survival functions for branching structures are distinct, it is also
essential to investigate how they differ by statistical tests and
dissimilarities calculation to validate LRWs further. Finally, section
~\ref{section:branch_conclusion} summarizes the demonstration and
verification of LRWs in the artificial images.


